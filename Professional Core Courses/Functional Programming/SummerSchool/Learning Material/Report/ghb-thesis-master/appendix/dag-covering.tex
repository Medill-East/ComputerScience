% Copyright (c) 2017-2018, Gabriel Hjort Blindell <ghb@kth.se>
%
% This work is licensed under a Creative Commons Attribution-NoDerivatives 4.0
% International License (see LICENSE file or visit
% <http://creativecommons.org/licenses/by-nc-nd/4.0/> for details).

\chapter{DAG Covering}
\labelAppendix{dag-covering}

This appendix considers techniques based on \gls{DAG covering}.
%
First, we introduce the principle in \refSection{dc-principle}.
%
We then prove in \refSection{dc-proof} that \gls{optimal.ps} \gls{pattern
  selection} using \glspl{DAG} is NP-complete.
%
We describe straightforward, greedy approaches in
\refSection{dc-greedy-techniques}, moving on to exhaustive techniques in
\refSection{dc-exhaustive-search-techniques}.
%
In \refSection{dc-extending-tree-covering-techniques-to-dags} we describe
techniques that extend methods from \gls{tree covering} to \glspl{DAG}.
%
\glsunset{MIS}%
\glsunset{MWIS}%
%
In \refSection{dc-mwis-based-approaches} we describe techniques that model
\gls{instruction selection} as a \gls{MIS} or \gls{MWIS} problem.
%
\glsreset{MIS}%
\glsreset{MWIS}%
%
In \refSection{dc-unate-binate-covering} we describe techniques that model
\gls{instruction selection} as a \glsshort{unate covering} or \gls{binate
  covering} problem.
%
In \refSectionList{dc-ip-based-approaches, dc-cp-based-approaches}, we describe
techniques based on methods from combinatorial optimization.
%
Other \gls{DAG}-based approaches that do not fit into any of the sections above
are discussed in \refSection{dc-other-approaches}.
%
Lastly, we discuss limitations of this principle in \refSection{dc-limitations}
and summarize in \refSection{dc-summary}.

The appendix is based on material presented in
\cite[Chap.\thinspace4]{HjortBlindell:2016:Survey} that has been adapted for
this dissertation.
%
To not disturb the flow of reading, material already presented in
\refChapter{existing-isel-techniques-and-reps} is duplicated in this appendix.


\section{The Principle}
\labelSection{dc-principle}

As we saw in \refAppendix{tree-covering}, the \gls{principle} of \gls{tree
  covering} has two significant disadvantages.
%
The first is that common subexpressions cannot be properly expressed in
\glspl{expression tree}, and the second is that many \glspl{instruction
  characteristic} -- such as \gls{multi-output.ic} \glspl{instruction} -- cannot
be modeled as \glspl{pattern tree}.
%
As these shortcomings are primarily due to the restricted use of \glspl{tree},
we can achieve a more powerful approach to \gls{instruction selection} by
operating \glspl{DAG}, thereby extending \gls{tree covering} to \gls{DAG
  covering}.

By lifting the restriction that every \gls{node} in the \gls{expression tree}
have exactly one \gls{parent}, we attain a \gls!{block DAG}.
%
Because \glspl{DAG} permit \glspl{node} to have multiple \glspl{parent}, the
intermediate values in an expression can be shared and reused within the same
\gls{block DAG}.
%
This also enables \glspl!{pattern DAG} that contain multiple \gls{root}
\glspl{node}, which signify the production of multiple output values.
%
Hence the \gls{instruction set} support is extended to include
\gls{multi-output.ic} \glspl{instruction}.

Since \glspl{DAG} are less restrictive compared to \glspl{tree}, transitioning
from \gls{tree covering} to \gls{DAG covering} requires new methods for solving
the problems of \gls{pattern matching} and \gls{pattern selection}.
%
\Gls{pattern matching} is typically addressed using one of the following
methods:
%
\begin{itemize}
  \item First split the \glspl{pattern DAG} into \glspl{tree}, then match these
    individually, and then recombine the matched \glspl{pattern tree} into their
    original \gls{DAG} form.
    %
    In general, matching \glspl{tree} on \glspl{DAG} is
    NP-complete~\cite{GareyJohnson:1979}, but designs applying this technique
    typically sacrifice completeness to retain linear time complexity.

  \item Match the \glspl{pattern DAG} directly using a generic \gls{subgraph
    isomorphism} algorithm.
    %
    Although such algorithms exhibit exponential worst-case time complexity, in
    the average case they often finish in polynomial time and are therefore used
    by several \gls{DAG covering}-based designs discussed in this appendix.
\end{itemize}

\Gls{optimal.ps} \gls{pattern selection} on \glspl{block DAG}, however, does not
offer the same range of choices in terms of complexity.


\section{Optimal Pattern Selection on DAGs Is NP-Complete}
\labelSection{dc-proof}

The cost of the gain in generality and modeling capabilities that \glspl{DAG}
give us is a substantial increase in complexity.
%
As we saw in \refAppendix{tree-covering}, selecting an \gls{optimal.ps} set of
\glspl{pattern} to cover a \gls{expression tree} can be done in linear time, but
doing the same for \glspl{block DAG} is an NP-complete problem.
%
Proofs were given in 1976 by \textcite{BrunoSethi:1976} and
\textcite{AhoEtAl:1976}, but these were most concerned with the optimality of
\gls{instruction scheduling} and \gls{register allocation}.
%
In 1995, \textcite{Proebsting:1995:Proof} gave a very concise proof for optimal
\gls{pattern selection}, and a longer, more detailed proof was given by
\textcite{KoesGoldstein:2008} in~2008.
%
In this dissertation, we will paraphrase the longer proof.


\subsection{The Proof}

The idea behind the proof is to transform the \gls{SAT} problem into an
\gls{optimal.ps} -- that is, \gls{least-cost.c} -- \gls{DAG covering} problem.
%
The \gls{SAT} problem is the task of deciding whether a Boolean formula, written
in \gls!{CNF}, can be satisfied.
%
A \gls{CNF} formula is an expression consisting of conjunctions of disjunctions
of Boolean \glspl{variable}
%
In other words, a formula is in \gls{CNF} if it has the following structure:
%
\begin{displaymath}
  (\mVar{x}_1 \mOr \mVar{x}_2 \mOr \ldots) \mAnd
  (\mVar{x}_{n + 1} \mOr \mVar{x}_{n + 2} \mOr \ldots) \mAnd
  \ldots
\end{displaymath}
%
where \mbox{$\mVar{x}_i \in \mSet{\text{true}, \text{false}}$} and $\mOr$ and
$\mAnd$ denotes logical-or and logical-and, respectively.
%
A \gls{variable}~$\mVar{x}$ can also be negated, written as \mbox{$\mNot
  \mVar{x}$}.

Since the \gls{SAT} problem is NP-complete, all polynomial-time transformations
from \gls{SAT} to any other problem $P$ must also render $P$ NP-complete.


\paragraph{Modeling SAT as a Covering Problem}

First, we transform an instance~$S$ of the \gls{SAT} problem into a \gls{block
  DAG}.
%
The goal is then to find an exact cover for the \gls{DAG} in order to deduce the
truth assignment for the Boolean \glspl{variable} from the set of
selected \glspl{pattern}.
%
For this purpose we will use $\mOr$, $\mAnd$, $\mNot$, $v$, $\mBox$, and
$\mStop$ as \gls{node} types, and define $\mType(n)$ as the type of a
\gls{node}~\mbox{$n$\hspace{-.8pt}.}
%
\Glspl{node} of type~$\mBox$ and $\mStop$ will be referred to as \glspl!{box
  node} and \glspl!{stop node}, respectively.
%
Now, for each Boolean \gls{variable}~\mbox{$x \in S$} we create two
\glspl{node}~$n_1$ and $n_2$ such that \mbox{$\mType(n_1) = v$} and
\mbox{$\mType(n_2) = \mBox$}, and add these to the \gls{block DAG}.
%
At the same time we also add an \gls{edge}~$\mEdge{n_1}{n_2}$.
%
The same is done for each binary Boolean operator~\mbox{$\mathit{op} \in S$} by
creating two \glspl{node}~$n'_1$ and $n'_2$ such that \mbox{$\mType(n'_1) = op$}
and \mbox{$\mType(n'_2) = \mBox$}, along with an
\gls{edge}~$\mEdge{n'_1}{n'_2}$.

To model the connection between the $\mathit{op}$ operation and its two input
operands~$x$ and~\mbox{$y$\hspace{-1pt},} we add two
\glspl{edge}~$\mEdge{n_x}{n'_1}$ and $\mEdge{n_y}{n'_1}$ such that
\mbox{$\mType(n_x) = \mType(n_y) = \mBox$.}
%
For the unary operation~$\neg$ we obviously only need one such \gls{edge}, and
since $\mOr$ and $\mAnd$ are commutative it does not matter in what order the
\glspl{edge} are arranged with respect to the operator \gls{node}.
%
Hence, in the resulting \gls{block DAG}, only \glspl{box node} will have more
than one \gls{outgoing.e} \gls{edge}.
%
An example of such a \gls{DAG} is shown in
\refFigure{sat-to-dag-covering-reduction}, which can be constructed in linear
time simply by traversing the Boolean formula.


\paragraph{Boolean Operations as Patterns}

\def\mPSat{P_{\mathsc{sat}}}

To cover the \gls{block DAG}, we will use the \glspl{pattern tree} given in
\refFigure{sat-patterns}, and we will refer to this \gls{pattern set}
as~$\mPSat$.
%
\begin{figure}
  \subcaptionbox{%
                  The SAT patterns.
                  %
                  For brevity, the patterns for the $\mAnd$~operation are
                  omitted as these can be easily inferred from the
                  $\mOr$~patterns.
                  %
                  All patterns are assumed to have the same unit cost%
                  \labelFigure{sat-patterns}%
                }{%
                  \begin{minipage}{87mm}%
                    \adjustbox{stack}{%
                      % Copyright (c) 2017-2018, Gabriel Hjort Blindell <ghb@kth.se>
%
% This work is licensed under a Creative Commons Attribution-NoDerivatives 4.0
% International License (see LICENSE file or visit
% <http://creativecommons.org/licenses/by-nc-nd/4.0/> for details).
%
\begingroup%
% Copyright (c) 2017-2018, Gabriel Hjort Blindell <ghb@kth.se>
%
% This work is licensed under a Creative Commons Attribution-NoDerivatives 4.0
% International License (see LICENSE file or visit
% <http://creativecommons.org/licenses/by-nc-nd/4.0/> for details).
%
\figureFont\figureFontSize%
\setlength{\nodeDist}{14pt}%
\setlength{\nodeSize}{12pt}%
\tikzset{%
  var node/.style={%
    node,
  },
  op node/.style={%
    node,
  },
  box node/.style={%
    value node,
  },
  stop node/.style={%
    node,
    regular polygon,
    regular polygon sides=6,
  },
  anchor node/.style={%
    node,
    minimum size=0.3\nodeSize,
    fill=black,
  },
}%
%
\begin{tikzpicture}
  \node [var node] (x) {$\mVar{x}$};
\end{tikzpicture}%
\endgroup%
\\
                      $\mVar{x} = F$
                    }%
                    \hfill%
                    \adjustbox{stack}{%
                      % Copyright (c) 2017-2018, Gabriel Hjort Blindell <ghb@kth.se>
%
% This work is licensed under a Creative Commons Attribution-NoDerivatives 4.0
% International License (see LICENSE file or visit
% <http://creativecommons.org/licenses/by-nc-nd/4.0/> for details).
%
\begingroup%
% Copyright (c) 2017-2018, Gabriel Hjort Blindell <ghb@kth.se>
%
% This work is licensed under a Creative Commons Attribution-NoDerivatives 4.0
% International License (see LICENSE file or visit
% <http://creativecommons.org/licenses/by-nc-nd/4.0/> for details).
%
\figureFont\figureFontSize%
\setlength{\nodeDist}{14pt}%
\setlength{\nodeSize}{12pt}%
\tikzset{%
  var node/.style={%
    node,
  },
  op node/.style={%
    node,
  },
  box node/.style={%
    value node,
  },
  stop node/.style={%
    node,
    regular polygon,
    regular polygon sides=6,
  },
  anchor node/.style={%
    node,
    minimum size=0.3\nodeSize,
    fill=black,
  },
}%
%
\begin{tikzpicture}
  \node [box node] (x box) {};
  \node [var node, below=of x box] (x) {$\mVar{x}$};
  \begin{scope}[data flow]
    \draw (x) -- (x box);
  \end{scope}
\end{tikzpicture}%
\endgroup%
\\
                      $\mVar{x} = T$
                    }%
                    \hfill%
                    \adjustbox{stack}{%
                      % Copyright (c) 2017-2018, Gabriel Hjort Blindell <ghb@kth.se>
%
% This work is licensed under a Creative Commons Attribution-NoDerivatives 4.0
% International License (see LICENSE file or visit
% <http://creativecommons.org/licenses/by-nc-nd/4.0/> for details).
%
\begingroup%
% Copyright (c) 2017-2018, Gabriel Hjort Blindell <ghb@kth.se>
%
% This work is licensed under a Creative Commons Attribution-NoDerivatives 4.0
% International License (see LICENSE file or visit
% <http://creativecommons.org/licenses/by-nc-nd/4.0/> for details).
%
\figureFont\figureFontSize%
\setlength{\nodeDist}{14pt}%
\setlength{\nodeSize}{12pt}%
\tikzset{%
  var node/.style={%
    node,
  },
  op node/.style={%
    node,
  },
  box node/.style={%
    value node,
  },
  stop node/.style={%
    node,
    regular polygon,
    regular polygon sides=6,
  },
  anchor node/.style={%
    node,
    minimum size=0.3\nodeSize,
    fill=black,
  },
}%
%
\begin{tikzpicture}
  \node [stop node] (stop) {};
  \node [box node, below=of stop] (box) {};
  \node [anchor node, below=of box] (anchor) {};
  \begin{scope}[data flow]
    \draw (anchor) -- (box);
    \draw (box) -- (stop);
  \end{scope}
\end{tikzpicture}%
\endgroup%
\\
                      satisfied
                    }%
                    \hfill%
                    \adjustbox{stack}{%
                      % Copyright (c) 2017-2018, Gabriel Hjort Blindell <ghb@kth.se>
%
% This work is licensed under a Creative Commons Attribution-NoDerivatives 4.0
% International License (see LICENSE file or visit
% <http://creativecommons.org/licenses/by-nc-nd/4.0/> for details).
%
\begingroup%
% Copyright (c) 2017-2018, Gabriel Hjort Blindell <ghb@kth.se>
%
% This work is licensed under a Creative Commons Attribution-NoDerivatives 4.0
% International License (see LICENSE file or visit
% <http://creativecommons.org/licenses/by-nc-nd/4.0/> for details).
%
\figureFont\figureFontSize%
\setlength{\nodeDist}{14pt}%
\setlength{\nodeSize}{12pt}%
\tikzset{%
  var node/.style={%
    node,
  },
  op node/.style={%
    node,
  },
  box node/.style={%
    value node,
  },
  stop node/.style={%
    node,
    regular polygon,
    regular polygon sides=6,
  },
  anchor node/.style={%
    node,
    minimum size=0.3\nodeSize,
    fill=black,
  },
}%
%
\begin{tikzpicture}
  \node [box node] (not box) {};
  \node [op node, below=of not box] (not) {$\mNot$};
  \node [box node, below=of not] (box) {};
  \node [anchor node, below=of box] (anchor) {};
  \begin{scope}[data flow]
    \draw (anchor) -- (box);
    \draw (box) -- (not);
    \draw (not) -- (not box);
  \end{scope}
\end{tikzpicture}%
\endgroup%
\\
                      $\mNot T = F$
                    }%
                    \hfill%
                    \adjustbox{stack}{%
                      % Copyright (c) 2017-2018, Gabriel Hjort Blindell <ghb@kth.se>
%
% This work is licensed under a Creative Commons Attribution-NoDerivatives 4.0
% International License (see LICENSE file or visit
% <http://creativecommons.org/licenses/by-nc-nd/4.0/> for details).
%
\begingroup%
% Copyright (c) 2017-2018, Gabriel Hjort Blindell <ghb@kth.se>
%
% This work is licensed under a Creative Commons Attribution-NoDerivatives 4.0
% International License (see LICENSE file or visit
% <http://creativecommons.org/licenses/by-nc-nd/4.0/> for details).
%
\figureFont\figureFontSize%
\setlength{\nodeDist}{14pt}%
\setlength{\nodeSize}{12pt}%
\tikzset{%
  var node/.style={%
    node,
  },
  op node/.style={%
    node,
  },
  box node/.style={%
    value node,
  },
  stop node/.style={%
    node,
    regular polygon,
    regular polygon sides=6,
  },
  anchor node/.style={%
    node,
    minimum size=0.3\nodeSize,
    fill=black,
  },
}%
%
\begin{tikzpicture}
  \node [op node] (not) {$\mNot$};
  \node [anchor node, below=of not] (anchor) {};
  \begin{scope}[data flow]
    \draw (anchor) -- (not);
  \end{scope}
\end{tikzpicture}%
\endgroup%
\\
                      $\mNot F = T$
                    }

                    \vspace{\betweensubfigures}

                    \adjustbox{stack}{%
                      % Copyright (c) 2017-2018, Gabriel Hjort Blindell <ghb@kth.se>
%
% This work is licensed under a Creative Commons Attribution-NoDerivatives 4.0
% International License (see LICENSE file or visit
% <http://creativecommons.org/licenses/by-nc-nd/4.0/> for details).
%
\begingroup%
% Copyright (c) 2017-2018, Gabriel Hjort Blindell <ghb@kth.se>
%
% This work is licensed under a Creative Commons Attribution-NoDerivatives 4.0
% International License (see LICENSE file or visit
% <http://creativecommons.org/licenses/by-nc-nd/4.0/> for details).
%
\figureFont\figureFontSize%
\setlength{\nodeDist}{14pt}%
\setlength{\nodeSize}{12pt}%
\tikzset{%
  var node/.style={%
    node,
  },
  op node/.style={%
    node,
  },
  box node/.style={%
    value node,
  },
  stop node/.style={%
    node,
    regular polygon,
    regular polygon sides=6,
  },
  anchor node/.style={%
    node,
    minimum size=0.3\nodeSize,
    fill=black,
  },
}%
%
\begin{tikzpicture}
  \node [op node] (or) {$\mOr$};
  \node [box node, position=-135 degrees from or] (box1) {};
  \node [box node, position=-45  degrees from or] (box2) {};
  \node [anchor node, below=of box1] (anchor1) {};
  \node [anchor node, below=of box2] (anchor2) {};
  \begin{scope}[data flow]
    \draw (anchor1) -- (box1);
    \draw (anchor2) -- (box2);
    \draw (box1) -- (or);
    \draw (box2) -- (or);
  \end{scope}
\end{tikzpicture}%
\endgroup%
\\
                      $T \mOr T = T$
                    }%
                    \hfill%
                    \adjustbox{stack}{%
                      % Copyright (c) 2017-2018, Gabriel Hjort Blindell <ghb@kth.se>
%
% This work is licensed under a Creative Commons Attribution-NoDerivatives 4.0
% International License (see LICENSE file or visit
% <http://creativecommons.org/licenses/by-nc-nd/4.0/> for details).
%
\begingroup%
% Copyright (c) 2017-2018, Gabriel Hjort Blindell <ghb@kth.se>
%
% This work is licensed under a Creative Commons Attribution-NoDerivatives 4.0
% International License (see LICENSE file or visit
% <http://creativecommons.org/licenses/by-nc-nd/4.0/> for details).
%
\figureFont\figureFontSize%
\setlength{\nodeDist}{14pt}%
\setlength{\nodeSize}{12pt}%
\tikzset{%
  var node/.style={%
    node,
  },
  op node/.style={%
    node,
  },
  box node/.style={%
    value node,
  },
  stop node/.style={%
    node,
    regular polygon,
    regular polygon sides=6,
  },
  anchor node/.style={%
    node,
    minimum size=0.3\nodeSize,
    fill=black,
  },
}%
%
\begin{tikzpicture}
  \node [op node] (or) {$\mOr$};
  \node [box node, position=-133 degrees from or] (box) {};
  \node [anchor node, position=-45 degrees from or] (anchor1) {};
  \node [anchor node, below=of box] (anchor2) {};
  \begin{scope}[data flow]
    \draw (anchor1) -- (or);
    \draw (anchor2) -- (box);
    \draw (box) -- (or);
  \end{scope}
\end{tikzpicture}%
\endgroup%
\\
                      $T \mOr F = T$
                    }%
                    \hfill%
                    \adjustbox{stack}{%
                      % Copyright (c) 2017-2018, Gabriel Hjort Blindell <ghb@kth.se>
%
% This work is licensed under a Creative Commons Attribution-NoDerivatives 4.0
% International License (see LICENSE file or visit
% <http://creativecommons.org/licenses/by-nc-nd/4.0/> for details).
%
\begingroup%
% Copyright (c) 2017-2018, Gabriel Hjort Blindell <ghb@kth.se>
%
% This work is licensed under a Creative Commons Attribution-NoDerivatives 4.0
% International License (see LICENSE file or visit
% <http://creativecommons.org/licenses/by-nc-nd/4.0/> for details).
%
\figureFont\figureFontSize%
\setlength{\nodeDist}{14pt}%
\setlength{\nodeSize}{12pt}%
\tikzset{%
  var node/.style={%
    node,
  },
  op node/.style={%
    node,
  },
  box node/.style={%
    value node,
  },
  stop node/.style={%
    node,
    regular polygon,
    regular polygon sides=6,
  },
  anchor node/.style={%
    node,
    minimum size=0.3\nodeSize,
    fill=black,
  },
}%
%
\begin{tikzpicture}
  \node [op node] (or) {$\mOr$};
  \node [anchor node, position=-135 degrees from or] (anchor1) {};
  \node [box node, position=-45 degrees from or] (box) {};
  \node [anchor node, below=of box] (anchor2) {};
  \begin{scope}[data flow]
    \draw (anchor1) -- (or);
    \draw (anchor2) -- (box);
    \draw (box) -- (or);
  \end{scope}
\end{tikzpicture}%
\endgroup%
\\
                      $F \mOr T = T$
                    }%
                    \hfill%
                    \adjustbox{stack}{%
                      % Copyright (c) 2017-2018, Gabriel Hjort Blindell <ghb@kth.se>
%
% This work is licensed under a Creative Commons Attribution-NoDerivatives 4.0
% International License (see LICENSE file or visit
% <http://creativecommons.org/licenses/by-nc-nd/4.0/> for details).
%
\begingroup%
% Copyright (c) 2017-2018, Gabriel Hjort Blindell <ghb@kth.se>
%
% This work is licensed under a Creative Commons Attribution-NoDerivatives 4.0
% International License (see LICENSE file or visit
% <http://creativecommons.org/licenses/by-nc-nd/4.0/> for details).
%
\figureFont\figureFontSize%
\setlength{\nodeDist}{14pt}%
\setlength{\nodeSize}{12pt}%
\tikzset{%
  var node/.style={%
    node,
  },
  op node/.style={%
    node,
  },
  box node/.style={%
    value node,
  },
  stop node/.style={%
    node,
    regular polygon,
    regular polygon sides=6,
  },
  anchor node/.style={%
    node,
    minimum size=0.3\nodeSize,
    fill=black,
  },
}%
%
\begin{tikzpicture}
  \node [box node] (or box) {};
  \node [op node, below=of or box] (or) {$\mOr$};
  \node [anchor node, position=-135  degrees from or] (anchor1) {};
  \node [anchor node, position=-45 degrees from or] (anchor2) {};
  \begin{scope}[data flow]
    \draw (anchor1) -- (or);
    \draw (anchor2) -- (or);
    \draw (or) -- (or box);
  \end{scope}
\end{tikzpicture}%
\endgroup%
\\
                      $F \mOr F = F$
                    }%
                  \end{minipage}%
                }%
  \hfill%
  \subcaptionbox{%
                  Example of a SAT problem represented as a DAG covering
                  problem%
                  \labelFigure{sat-instance}%
                }%
                [34mm]%
                {%
                  \adjustbox{stack}{%
                    % Copyright (c) 2017-2018, Gabriel Hjort Blindell <ghb@kth.se>
%
% This work is licensed under a Creative Commons Attribution-NoDerivatives 4.0
% International License (see LICENSE file or visit
% <http://creativecommons.org/licenses/by-nc-nd/4.0/> for details).
%
\begingroup%
% Copyright (c) 2017-2018, Gabriel Hjort Blindell <ghb@kth.se>
%
% This work is licensed under a Creative Commons Attribution-NoDerivatives 4.0
% International License (see LICENSE file or visit
% <http://creativecommons.org/licenses/by-nc-nd/4.0/> for details).
%
\figureFont\figureFontSize%
\setlength{\nodeDist}{14pt}%
\setlength{\nodeSize}{12pt}%
\tikzset{%
  var node/.style={%
    node,
  },
  op node/.style={%
    node,
  },
  box node/.style={%
    value node,
  },
  stop node/.style={%
    node,
    regular polygon,
    regular polygon sides=6,
  },
  anchor node/.style={%
    node,
    minimum size=0.3\nodeSize,
    fill=black,
  },
}%
%
\pgfdeclarelayer{background}%
\pgfsetlayers{background,main}%
\begin{tikzpicture}[
    match fill/.style={
      fill=shade2,
    },
    box match node/.style={
      match node,
      inner sep=2.8pt,
    },
    stop match node/.style={
      match node,
      inner sep=1.4pt,
    },
  ]

  % DAG
  \node [stop node] (stop) {};
  \node [box node, below=of stop] (and box) {};
  \node [op node, below=of and box] (and) {$\mAnd$};
  \node [box node, position=-135 degrees from and] (or box) {};
  \node [box node, position=- 45 degrees from and] (not box) {};
  \node [op node, below=of or box] (or) {$\mOr$};
  \node [op node, below=of not box] (not) {$\mNot$};
  \node [box node, position=-135 degrees from or] (x1 box) {};
  \node [box node, position=- 45 degrees from or] (x2 box) {};
  \node [var node, below=of x1 box] (x1) {$\mVar{x}_1$};
  \node [var node, below=of x2 box] (x2) {$\mVar{x}_2$};
  \begin{scope}[data flow]
    \draw (x1) -- (x1 box);
    \draw (x2) -- (x2 box);
    \draw (x1 box) -- (or);
    \draw (x2 box) -- (or);
    \draw (x2 box) -- (not);
    \draw (or) -- (or box);
    \draw (not) -- (not box);
    \draw (or box) -- (and);
    \draw (not box) -- (and);
    \draw (and) -- (and box);
    \draw (and box) -- (stop);
  \end{scope}

  % Matches
  \begin{pgfonlayer}{background}
    \node [match node, fit=(x1), match fill] {};
    \node [match node, fit=(not), match fill] {};

    \begin{scope}[match line]
      \node [match node, draw=none, fill=none, fit=(x2)] (ma) {};
      \node [box match node, draw=none, fill=none, fit=(x2 box)] (mb) {};
      \draw [match line, bend left=45, match fill]
            (ma.west)
            --
            (mb.west)
            to
            (mb.north)
            to
            (mb.east)
            --
            (ma.east)
            to
            (ma.south)
            to
            cycle;

      \node [box match node, draw=none, fill=none, fit=(x1 box)] (ma) {};
      \node [match node, draw=none, fill=none, fit=(or)] (mb) {};
      \draw [match line, bend left=45, match fill]
            (ma.south east)
            to
            (ma.south west)
            to
            (ma.north west)
            --
            (mb.north west)
            to
            (mb.north east)
            to
            (mb.south east)
            --
            cycle;

      \node [box match node, draw=none, fill=none, fit=(or box)] (ma) {};
      \node [match node, draw=none, fill=none, fit=(and)] (mb) {};
      \node [box match node, draw=none, fill=none, fit=(not box)] (mc) {};
      \draw [match line, bend left=45, match fill]
            (ma.north west)
            --
            (mb.north west)
            to
            (mb.north east)
            --
            (mc.north east)
            to
            (mc.south east)
            to [bend angle=90]
            (mc.south)
            --
            (ma.south)
            to [bend angle=90]
            (ma.south west)
            to
            cycle;

      \node [box match node, draw=none, fill=none, fit=(and box)] (ma) {};
      \node [stop match node, draw=none, fill=none, fit=(stop)] (mb) {};
      \draw [match line, bend left=45, match fill]
            (ma.west)
            --
            (mb.west)
            to
            (mb.north)
            to
            (mb.east)
            --
            (ma.east)
            to
            (ma.south)
            to
            (ma.west);
    \end{scope}
  \end{pgfonlayer}
\end{tikzpicture}%
\endgroup%
\\
                    $(\mVar{x}_1 \mOr \mVar{x}_2) \mAnd (\mNot \mVar{x}_2)$
                  }%
                }

  \caption[Transforming SAT into DAG covering]
          {Transforming SAT to DAG covering~\cite{KoesGoldstein:2008}}
  \labelFigure{sat-to-dag-covering-reduction}
\end{figure}
%
Every \gls{pattern} in $\mPSat$ adheres to the following invariant:
%
\begin{enumerate}
  \item If a \gls{variable}~$\mVar{x}$ is set to true ($T$), then the selected
    \gls{pattern} covering the $\mVar{x}$~\gls{node} will also cover the
    corresponding \gls{box node} of~$\mVar{x}$.
  \item If the result of an operation $\mathit{op}$ evaluates to false ($F$),
    then that \gls{pattern} will not cover the corresponding \gls{box node}
    of~$\mathit{op}$.
\end{enumerate}
%
Another way of looking at it is that an operator in a \gls{pattern}
\emph{consumes} a \gls{box node} if its corresponding value must be set
to~$T$\!, and \emph{produces} a \gls{box node} if the result must evaluate
to~\mbox{$F$\hspace{-1pt}.}
%
Using this scheme, we can easily deduce the truth assignments to the
\glspl{variable} by inspecting whether the \glspl{pattern} selected to cover the
\gls{DAG} consume the \glspl{box node} of the \glspl{variable}.
%
Since the only \gls{pattern} to contain a \gls{stop node} also consumes a
\gls{box node}, the entire expression will be forced to evaluate to~$T$\!.

In addition to the \gls{node} types that can appear in the \gls{block DAG}, the
\glspl{pattern} can also contain \glspl{node} of an additional type, $\mAnchor$,
which we will refer to as \glspl!{anchor node}.
%
Let $\mNumChildren(n)$ denote the number of \glspl{child}
of~\mbox{$n$\hspace{-.8pt},} and \mbox{$\mChild(i\hspace{-.8pt}, n)$} the $i$th
\gls{child} of~\mbox{$n$\hspace{-.8pt}.}
%
We now say that a \gls{pattern}~\mbox{$p$\hspace{-.8pt},} with \gls{root}
\gls{node}~$p_r$, \emph{matches} the part of a \gls{block
  DAG}~\mbox{$\mTuple{N\hspace{-1pt}, E}$} which is rooted at a
\gls{node}~\mbox{$n \in N$} if and only if:
%
\begin{enumerate}
  \item $\mType(n) = \mType(p_r)$,
  \item $\mNumChildren(n) = \mNumChildren(p_r)$, and
  \item $\forall 1 \leq i \leq \mNumChildren(n) :
    \mType(\mChild(i\hspace{-.8pt}, n)) = \mAnchor \mOr
    \text{$\mChild(i\hspace{-.8pt}, n)$ matches $\mChild(i\hspace{-.8pt},
      p_r)$}$.
\end{enumerate}
%
In other words, the structure of the \gls{pattern tree} -- which includes the
\gls{node} types and \glspl{edge} -- must correspond to the structure of the
matched \gls{subgraph} (excluding \glspl{anchor node}, which can match any
\gls{node} in the \gls{block DAG}).

We introduce several new definitions.
%
Given a \gls{block DAG}~\mbox{$G = \mTuple{N\hspace{-1pt}, E}$}, let
$\mMatchSet[n]$ be the set of \glspl{pattern} in $\mPSat$ that match at
\gls{node}~\mbox{$n \in N$\hspace{-1pt}.}
%
Also, given a \gls{pattern}~\mbox{$\mTuple{N_p, E_p}$}, let \mbox{$\mMatched(p,
  n_p)$} be the set of \glspl{node} in $N$ that are matched by a \gls{node}~$n_p
\in N_p$.
%
Lastly, we say that $G$ is \emph{covered} by a function~\mbox{$f : N \rightarrow
  2^{\mPSat}$}, which maps \glspl{node} in the \gls{block DAG} to a set of
\glspl{pattern}, if and only if, for each \mbox{$n \in N$\hspace{-1pt},}
%
\begin{enumerate}
  \item $\forall p \in f(n) : \text{$p$ matches $n$}$\hspace{-.8pt},
  \item $\mType(n) = \mStop \mImp f(n) \neq \emptyset$, and
  \item $\forall p = \mTuple{N_p, E_p} \in f(v), n_p \in N_p :
    \mType(n_p) = \mAnchor \mImp f(\mMatched(n, n_p)) \neq
    \emptyset$.
\end{enumerate}
%
The first \gls{constraint} enforces that only valid \glspl{match} are selected.
%
The second \gls{constraint} enforces that some \gls{match} has been selected to
cover the \gls{stop node}.
%
The third \gls{constraint} enforces that \glspl{match} have been selected to
cover the rest of the \gls{DAG}.
%
An \gls{optimal.ps} cover is thus a mapping~$f$ which covers the \gls{block
  DAG}~\mbox{$\mTuple{N\hspace{-1pt}, E}$} and also minimize
%
\begin{displaymath}
  \sum_{\mathclap{n \, \in \, N}}
  \hspace{.8em}
  \sum_{\mathclap{p \, \in \, f(n)}}
  \mCost(p),
\end{displaymath}
%
where $\mCost(p)$ is the cost of \gls{pattern}~\mbox{$p$\hspace{-.8pt}.}


\paragraph{Optimal Solution to DAG Covering $\Rightarrow$ Solution to SAT}

We now postulate that if the \gls{optimal.ps} cover has a total cost equal to
the number of non-\glspl{box node} in the~\gls{block DAG}, then the
corresponding \gls{SAT} problem is satisfiable.
%
Since all \glspl{pattern} in $\mPSat$ cover exactly one non-\gls{box node} and
have equal unit cost, the condition above means that every \gls{node} in the
\gls{DAG} is covered by exactly one \gls{pattern}.
%
This in turn means that exactly one value will be assumed for every Boolean
\gls{variable} and operator result, which is easy to deduce through inspection
of the selected \glspl{match}.

We have thereby shown that an instance of the \gls{SAT} problem can be solved by
transforming it, in polynomial time, to an instance of the \gls{optimal.ps}
\gls{DAG covering} problem.
%
Hence \gls{optimal.ps} \gls{DAG covering} -- and therefore also optimal
\gls{instruction selection} based on \gls{DAG covering} -- is NP-complete.
%
\hfill\qedsymbol


\section{Straightforward, Greedy Techniques}
\labelSection{dc-greedy-techniques}

Since \gls{instruction selection} on \glspl{DAG} with \gls{optimal.ps}
\gls{pattern selection} is computationally difficult, most \glspl{instruction
  selector} based on this \gls{principle} are suboptimal.
%
One of the first \glspl{code generator} to operate on \glspl{DAG} was developed
by \textcite{AhoEtAl:1976}.
%
In their 1976~paper, \citeauthor{AhoEtAl:1976} introduce some simple greedy
heuristics for producing \gls{assembly code} for a commutative
one-\gls{register} \gls{target machine}.
%
However, these methods assume a \mbox{1-to-1} mapping between the \glspl{node}
in a \gls{block DAG} and the \glspl{instruction} and thus effectively ignore the
\gls{instruction selection} problem.


\subsection{LLVM}
\labelSection{llvm}

A more flexible, but still greedy, heuristic is applied in the well-known
\gls!{LLVM} \gls{compiler} infrastructure~\cite{LattnerAdve:2004}.
%
According to a blog entry by \textcite{Bendersky:2013} -- which at the time of
writing provides the only documentation, except for the source code itself --
the \gls{instruction selector} is basically a greedy
\mbox{\gls{DAG}-to-\gls{DAG}} rewriter.\!%
%
\footnote{%
  \gls{LLVM} is also equipped with a ``fast'' \gls{instruction selector}, but it
  is implemented as a typical \gls{macro expander} and is only intended to be
  used when compiling without extensive \gls{program} optimization.%
}

The \glspl{pattern} -- which are limited to \glspl{tree} -- are expressed in a
\gls{machine description} that allows common features to be factored out into
abstract \glspl{instruction}.
%
A tool called \gls!{TableGen} expands the abstract \glspl{instruction} into
\glspl{pattern tree}, which are then processed by a matcher generator.
%
To ensure a partial order among all \glspl{pattern}, the matcher generator first
performs a lexicographical sort on the \gls{pattern set}, in the following
order:
%
\begin{inlinelist}[label=(\roman*), itemjoin={;\ }, itemjoin*={; and\ }]
  \item by decreasing complexity, which is the sum of the \gls{pattern}'s size
    and a constant that can be tweaked to give higher priority for particular
    \glspl{instruction}
  \item by increasing cost
  \item by increasing size of the \gls{subgraph} that replaces the covered part
    in the \gls{block DAG} (if the corresponding \gls{pattern} is selected)
\end{inlinelist}.
%
Once sorted, the \glspl{pattern} are converted into small recursive
\glspl{program} which essentially check whether the corresponding \gls{pattern}
matches at a given \gls{node} in the \gls{block DAG}.
%
These \glspl{program} are then compiled into a form of byte code and assembled
into a matcher table, arranged such that the lexicographical sort is preserved.
%
The \gls{instruction selector} applies this table by simply executing the byte
code, starting with the first element.
%
When a \gls{match} is found, the \gls{pattern} is greedily selected and the
matched \gls{subgraph} is replaced with the output (usually a single \gls{node})
of the selected \gls{pattern}.
%
This process repeats until there are no \glspl{node} remaining in the original
\gls{block DAG}.

Although in extensive use (as of \mbox{version 3.4}), \gls{LLVM}'s
\gls{instruction selector} has several drawbacks.
%
The main disadvantage is that any \gls{pattern} that is not supported by
\gls{TableGen} has to be handled manually through custom \gls{C}~functions.
%
Unlike \gls{GCC} -- which applies \gls{macro expansion} combined with
\gls{peephole optimization} (see
\refSection{macro-expansion-with-peephole-optimization}) -- this includes all
\gls{multi-output.ic} \glspl{instruction}, since \gls{LLVM} is restricted to
\glspl{pattern tree} only.
%
In addition, the greedy scheme compromises code quality.


\section{Techniques Based on Exhaustive Search}
\labelSection{dc-exhaustive-search-techniques}

Although \gls{optimal.ps} \gls{pattern selection} can be achieved through
exhaustive search, in practice this is typically infeasible due to the
exponential number of possible combinations.
%
Nonetheless, there do exist a few techniques that do exactly this, but they
apply various techniques to prune the search space.


\subsection{Extending Means-End Analysis to DAGs}

Twenty years after \citeauthor{Newcomer:1975} and \citeauthor{CattellEtAl:1979}
(see \refSection{tree-covering-first-approaches}),
\citeauthor{YuHu:1994a}~\cite{YuHu:1994a, YuHu:1994b} rediscovered
\gls{means-end analysis} as a method for \gls{instruction selection}.
%
They also made two major improvements.
%
First, \citeauthor{YuHu:1994a}'s design supports \glsshort{block DAG} and
\glspl{pattern DAG} whereas those by~\citeauthor{Newcomer:1975} and
Cattell~\etal are both limited to \glspl{tree}.
%
Second, it combines \gls{means-end analysis} with \gls!{hierarchical
  planning}~\cite{Sacerdoti:1973}, which is a \gls{search} strategy that relies
on the fact that many problems can be arranged in a hierarchical manner for
handling larger and more complex problem instances.
%
Using \gls{hierarchical planning} enables exhaustive exploration of the
\gls{search space} while at the same time avoiding the situations of dead ends
and infinite looping that may occur in straightforward implementations of
\gls{means-end analysis} (\citeauthor{Newcomer:1975} and Cattell~\etal both
circumvented this problem by enforcing a cut-off when a certain depth in the
\gls{search space} had been reached).

Although this technique exhibits a worst time execution that is exponential in
the search depth, \citeauthor{YuHu:1994a} assert that a depth of~\num{3} is
sufficient to yield results of equal quality to that of handwritten
\gls{assembly code}.
%
This claim notwithstanding, it is unclear whether it can be extended to support
complex \glspl{instruction} such as \gls{inter-block.ic} and
\gls{interdependent.ic} \glspl{instruction}.


\subsection{Relying on Semantic-Preserving Transformations}

In 1996, \textcite{HooverZadeck:1996} developed a system called \gls!{Toast}
with the goal of automating the generation of entire \gls{compiler} frameworks
-- including \gls{instruction scheduling} and \gls{register allocation} -- from
a declarative \gls{machine description}.
%
In \gls{Toast} the \gls{instruction selection} is done by applying
semantic-preserving transformations during \gls{pattern selection} to make
better use of the \gls{instruction set}.
%
For example, although \mbox{$x * 2$} is semantically equivalent to \mbox{$x \ll
  1$}, where $x$ is arithmetically shifted one bit to the right which is a
faster computation than multiplication.
%
Most \glspl{instruction selector}, however, will fail to select
\glspl{instruction} implementing the latter when the former appears in the
\gls{block DAG} as the \glspl{pattern} are syntactically different from one
another.

Their design works as follows.
%
First, the \gls{frontend} emits \glspl{block DAG} consisting of \glspl{semantic
  primitive}, a kind of \gls{IR} code also used for describing the
\glspl{instruction}.
%
The \gls{block DAG} is then semantically matched using \gls{single-output.ic}
\glspl{pattern} derived from the \glspl{instruction}.
%
Semantic \glspl{match} -- which \citeauthor{HooverZadeck:1996} call \glspl!{toe
  print} -- and are found by a \gls{semantic comparator}.
%
The \gls!{semantic comparator} first performs syntactic matching -- that is,
checking that the \glspl{node} are of the same type, which is done using a
straightforward \mbox{$\mBigO(nm)$}~algorithm -- and then resorts to
semantic-preserving transformations for when syntactic matching fails.
%
To bound the exhaustive search for all possible \glspl{toe print}, a
transformation is only applied if it will lead to a syntactic \gls{match} later
on.
%
Once all \glspl{toe print} have been found, they are combined into \glspl!{foot
  print}, which correspond to the full effects of an \gls{instruction}.
%
A \gls{foot print} can consist of just a single \gls{toe print} (as with
\gls{single-output.ic} \glspl{instruction}) or several (as with
\gls{multi-output.ic} \glspl{instruction}), but the paper lacks details on how
this is done exactly.
%
Lastly, all combinations of \glspl{foot print} are considered in pursuit of the
one leading to the most effective implementation of the \gls{block DAG}.
%
To further prune the \gls{search space}, this process only considers
combinations where each selected \gls{foot print} syntactically matches at least
one \gls{semantic primitive} in the \gls{block DAG}.
%
In addition, only ``trivial amounts'' of the \gls{block DAG} (such as
\glspl{node} representing constants) may be included in more than one \gls{foot
  print}.

Using a prototype implementation, \citeauthor{HooverZadeck:1996} reported that
almost $10^{70}$~``implied \gls{instruction} \glspl{match}'' were found for one
of the test cases, but it is unclear how many of them were actually useful.
%
Moreover, in its current form the design appears to be unpractical for
generating \gls{assembly code} for all but very small \glspl{function}.


\section{Extending Tree Covering Techniques to DAGs}
\labelSection{dc-extending-tree-covering-techniques-to-dags}

Another common approach to \gls{DAG covering} is to reuse already-known,
linear-time methods from \gls{tree covering}.
%
This can be achieved either by transforming the \glspl{block DAG} into
\glspl{tree}, or by generalizing the \gls{tree}-based algorithms for
\gls{pattern matching} and \gls{pattern selection}.
%
We begin by discussing designs that apply the first technique.


\subsection{Undagging Block DAGs}

The simplest approach for reusing \gls{tree covering} techniques is to transform
the \gls{block DAG} into several \glspl{expression tree}.
%
We will refer to this idea as \gls!{undagging}.

As illustrated in \refFigure{undagging-example}, a \gls{block DAG} can be
\glsshort{undagging}[ged] into \glspl{expression tree} in two ways.
%
\begin{figure}
  \subcaptionbox{%
                  Block DAG with matches%
                  \labelFigure{undagging-example-original}%
                }%
                [20mm]%
                {%
                  % Copyright (c) 2017-2018, Gabriel Hjort Blindell <ghb@kth.se>
%
% This work is licensed under a Creative Commons Attribution-NoDerivatives 4.0
% International License (see LICENSE file or visit
% <http://creativecommons.org/licenses/by-nc-nd/4.0/> for details).
%
\begingroup%
% Copyright (c) 2017-2018, Gabriel Hjort Blindell <ghb@kth.se>
%
% This work is licensed under a Creative Commons Attribution-NoDerivatives 4.0
% International License (see LICENSE file or visit
% <http://creativecommons.org/licenses/by-nc-nd/4.0/> for details).
%
\figureFont\figureFontSize%
\setlength{\nodeDist}{14pt}%
\setlength{\nodeSize}{12pt}%
\tikzset{%
  label/.style={
    nothing,
    node distance=1.5mm,
    font=\relsize{-.5},
  },
}%
%
\begin{tikzpicture}
  \node [node] (add) {\nAdd};
  \node [node, below=of add] (mul) {\nMul};
  \node [node, position=-135 degrees from mul] (x) {\nVar{x}};
  \node [node, position=- 45 degrees from mul] (y) {\nVar{y}};
  \begin{scope}[data flow, bend angle=50]
    \draw (x) -- (mul);
    \draw (y) -- (mul);
    \draw (mul) to [bend right] (add);
    \draw (mul) to [bend left]  (add);
  \end{scope}
\end{tikzpicture}%
\endgroup%
%
                }%
  \hfill%
  \subcaptionbox{%
                  After edge splitting%
                  \labelFigure{undagging-example-after-splitting}%
                }%
                {%
                  \input{%
                    figures/dag-covering/undagging-example-after-splitting%
                  }%
                }%
  \hfill%
  \subcaptionbox{%
                  After node duplication%
                  \labelFigure{undagging-example-after-duplication}%
                }%
                {%
                  \input{%
                    figures/dag-covering/undagging-example-after-duplication%
                  }%
                }%

  \caption{Example of undagging a block DAG}
  \labelFigure{undagging-example}
\end{figure}
%
The first approach is to split the edges involving shared \glspl{node} -- these
are \glspl{node} where reuse occurs due to the presence of common subexpressions
-- which results in a set of disconnected \glspl{expression tree} that can then
be covered individually.
%
Not surprisingly, this approach is called \gls!{edge splitting}.
%
An implicit connection between the \glspl{expression tree} is maintained by
forcing the values computed at the shared \glspl{node} to be stored and read
from a specific location, typically in memory.
%
An example of such an implementation is \gls!{Dagon}, a technology binder
developed by \textcite{Keutzer:1987}, which maps technology-independent
descriptions onto circuits.
%
The second approach is to duplicate the \glspl{node} involved in computing the
shared value, which is known as \gls!{node duplication}.
%
This results in a single but larger \gls{expression tree} compared to those
produced with \gls{edge splitting}.

Common for both schemes is that they compromise code quality:
%
\begin{inlinelist}[itemjoin={; }, itemjoin*={; and}]
  \item too aggressive \gls{edge splitting} produces many small \glspl{tree}
    that cannot be covered using larger \glspl{pattern}, inhibiting use of more
    efficient \glspl{instruction}
  \item too aggressive \gls{node duplication} incurs a larger computational
    workload where many operations are needlessly re-executed in the final
    \gls{assembly code}
\end{inlinelist}.
%
Moreover, the intermediate results of an edge-split \gls{block DAG} must be
forcibly stored in specific locations, which can be troublesome for
heterogeneous memory-\gls{register} architectures (this particular problem was
studied by \textcite{AraujoEtAl:1996}).


\paragraph{Balancing Splitting and Duplication}

In 1994, \citeauthor{FauthEtAl:1994}~\cite{FauthEtAl:1994, Muller:1994}
developed a technique that tries to mitigate the deficiencies of \gls{undagging}
by balancing the use of \gls{node duplication} and \gls{edge splitting}.
%
Implemented in the \gls!{CBC}, the \gls{instruction selector} applies a
heuristic algorithm that first favors \gls{node duplication}, and resorts to
\gls{edge splitting} when the former is deemed too costly.
%
The decision about whether to duplicate or to split is taken by comparing the
cost of the two solutions and selecting the cheapest one.
%
The cost is calculated as a weighted sum \mbox{$w_1 n_{\text{dup}} + w_2
  n_{\text{split}}$}, where $n_{\text{dup}}$ is the number of \glspl{node} in
the \gls{block DAG} (a rough estimate of code size), and $n_{\text{split}}$ is
the expected number of \glspl{node} executed along each execution path (a rough
estimate of execution time).
%
Once this is done, each resulting \gls{expression tree} is covered by an
improved version of \gls{IBurg} (see \refAppendix{tree-covering} on
\refPageOfSection{tc-iburg}) with extended match condition support.
%
However, the experimental data is too limited for us to judge how efficient this
technique is compared to a design where the \glspl{block DAG} have been
transformed into \glspl{expression tree} using just one method.


\paragraph{Register-Sensitive Instruction Selection}

In 2001, \textcite{SarkarEtAl:2001} developed an \gls{instruction selection}
technique that attempts to reduce the \gls!{register pressure} -- that is, the
number of \glspl{register} needed by the \gls{function} -- in order to
facilitate \gls{register allocation}.\!%
%
\footnote{%
  Another register-aware \gls{instruction selection} technique was developed in
  2014 by \textcite{XieEtAl:2014}, with the aim of reducing the number of writes
  to a nonvolatile \gls{register} file.
  %
  However, the \glspl{instruction} are selected using a proprietary and greedy
  heuristic hat does not warrant in-depth discussion.%
}

The design works as follows.
%
The \gls{block DAG} is first augmented with additional \glspl{edge} to signify
scheduling dependencies between memory operations, and then it is split into a
several \glspl{expression tree} using a heuristic to decide which \glspl{edge}
to break.
%
The \glspl{expression tree} are then covered individually using conventional
methods based on \gls{tree covering}.
%
However, instead of being the usual number of execution cycles, the cost of each
\gls{instruction} is set so as to reflect the amount of \gls{register pressure}
incurred by that instruction (unfortunately, the paper lacks details on how
these costs are computed exactly).
%
Once \glspl{pattern} have been selected, the \glspl{node} which are covered by
the same \gls{pattern} are merged into \glspl{super node}.
%
The resulting \gls{graph} is then checked for whether it contains any
\glspl{cycle}, which may appear due to the extra data dependencies that were
added at the earlier stage.
%
If it does, it means that there exist cyclic scheduling dependencies between two
or more memory operations, making it an illegal cover.
%
The splits are then reverted and the process repeats until a legal cover is
found.

\citeauthor{SarkarEtAl:2001} implemented their \gls{register}-sensitive design
in \gls!{Jalapeno}, a \gls{register}-based \gls{Java} virtual machine developed
by \gls{IBM}.
%
For a small set of problems the performance increased by about
\SI{10}{\percent}, which \citeauthor{SarkarEtAl:2001} claim to be due to fewer
\glspl{instruction} needed for \gls{register} \gls{spilling.r} compared to the
default \gls{instruction selector}.
%
Although innovative, it is doubtful that the technique can be extended much
further.


\subsection{Extending the Dynamic Programming Approach to DAGs}

To avoid the application of ad hoc heuristics, several \gls{DAG}-based
\glspl{instruction selector} perform \gls{pattern selection} by applying an
extension of the \gls{tree}-based \gls{DP}~algorithm originally developed by
\textcite{AhoJohnson:1976}.
%
According to the literature, \citeauthor{LiemEtAl:1994}~\cite{LiemEtAl:1994,
  PaulinEtAl:1994, PaulinEtAl:1995} appear to have been the first to have done
so.

In a seminal paper from~1994, \citeauthor{LiemEtAl:1994} introduce a design
which is part of \gls!{CodeSyn}, a well-known code synthesis system, which in
turn is part of a development environment for embedded systems called
\gls!{FlexWare}.
%
For \gls{pattern matching}, \citeauthor{LiemEtAl:1994} applied the same
technique as \textcite{Weingart:1973} (see
\refSection{first-techniques-to-use-tree-pattern-matching}) by combining all
available \glspl{pattern tree} into a single \gls{tree} of \glspl{pattern}.
%
This \gls{pattern tree} is traversed in tandem with the \gls{block DAG}, and for
each \gls{node} an \mbox{$\mBigO(nm)$} \gls{pattern matcher} is used to find all
\glspl{match set}.
%
\Gls{pattern selection} is then performed using an extended version of the
\gls{DP}~algorithm, but the paper does not explain how this is done exactly.
%
Moreover, the algorithm is only applied on the data flow of the \gls{block DAG}
-- control flow is covered separately using a simple heuristic -- and no
guarantees are made that the \gls{pattern selection} is \gls{optimal.ps}, as
that is an NP-complete problem.


\paragraph{Potentially Optimal Pattern Selection}

In a paper from~1999, \textcite{Ertl:1999} introduces a design which guarantees
\gls{optimal.ps} \gls{pattern selection} on \glspl{block DAG} for certain
\glspl{machine grammar}.
%
The idea is to first make a bottom-up pass over the \gls{block DAG} and compute
the costs using the conventional \gls{DP}~algorithm as discussed in
\refAppendix{tree-covering}.
%
Each \gls{node} is thus labeled with the same costs, as if the \gls{block DAG}
had first been transformed into a \gls{tree} through \gls{node} duplication.
%
But \citeauthor{Ertl:1999} recognized that if several \glspl{pattern} reduce the
same \gls{node} to the same \gls{nonterminal}, then the reduction to that
\gls{nonterminal} can be shared between several \glspl{rule} whose
\glspl{pattern} contain the \gls{nonterminal}.
%
Hence the \glspl{instruction} for implementing shared \glspl{nonterminal} only
need to be emitted once, decreasing code size and also improving performance,
since the amount of redundant computation is reduced.
%
With appropriate data structures, a linear-time implementation can be achieved.

An example illustrating such a situation is given in \refFigure{ertl-example},
where we see an addition that will have to be implemented twice, as its
\gls{node} is covered by two separate \glspl{pattern} each of which reduces the
\gls{subtree} to a different \gls{nonterminal}.
%
\begin{figure}
  \mbox{}%
  \hfill%
  \adjustbox{valign=M}{%
    % Copyright (c) 2017-2018, Gabriel Hjort Blindell <ghb@kth.se>
%
% This work is licensed under a Creative Commons Attribution-NoDerivatives 4.0
% International License (see LICENSE file or visit
% <http://creativecommons.org/licenses/by-nc-nd/4.0/> for details).
%
\begingroup%
\figureFont\figureFontSize%
\setlength{\nodeDist}{28pt}%
%\setlength{\nodeSize}{12pt}%
\pgfdeclarelayer{b2}%
\pgfdeclarelayer{b1}%
\pgfsetlayers{b2,b1,main}%
\colorlet{m1shade}{shade1}%
\colorlet{m2shade}{shade2!25!shade3}%
\begin{tikzpicture}[
    reduction label/.style={
      nothing,
      inner xsep=2pt,
      auto,
    },
    outer match node/.style={
      match node,
      inner sep=-1pt,
    },
  ]

  % DAG
  \node [node] (n1) {\nMul};
  \node [node, position=-130 degrees from n1] (n2) {\nLoad};
  \node [node, position=-50  degrees from n2] (n3) {\nAdd};
  \node [node, position=-130 degrees from n3] (n4) {\nVar{4}};
  \node [node, position=-50  degrees from n3] (n5) {\nVar{a}};
  \begin{scope}[data flow]
    \draw (n5)
          -- node [reduction label, swap] {$\mNT{Reg}, \mNT{Reg}$}
          (n3);
    \draw (n4)
          -- node [reduction label, auto=none, above, sloped, inner ysep=2pt]
                  {$\mNT{Reg}$}
          (n3);
    \draw (n3)
          -- node [reduction label] {$\mNT{Addr}$}
          (n2);
    \draw (n3)
          to [bend right=45]
          node [reduction label, swap, pos=.26] {$\mNT{Reg}$}
          (n1);
    \draw (n2)
          -- node [reduction label, inner xsep=1pt] {$\mNT{Reg}$}
          (n1);
  \end{scope}

  % Matches
  \begin{pgfonlayer}{b1}
    \node [match node, fit=(n1), shading=axis, shading angle=90,
           left color=m2shade, right color=m1shade]
          (m1) {};
    \node [match node, fit=(n2), fill=m2shade] (m2) {};
    \node [match node, fit=(n3), fill=m1shade] (m3) {};
    \node [match node, fit=(n4), fill=m1shade] (m4) {};
    \node [match node, fit=(n5), fill=m1shade] (m5) {};
  \end{pgfonlayer}
  \begin{pgfonlayer}{b2}
    \node [outer match node, fit=(m5), fill=m2shade] (m5) {};

    \node [outer match node, fit=(m4), draw=none] (m5a) {};
    \node [outer match node, fit=(m3), draw=none] (m5b) {};
    \draw [match line, bend left=45, fill=m2shade]
          (m5a.south east)
          to
          (m5a.south west)
          to
          (m5a.north west)
          --
          (m5b.north west)
          to
          (m5b.north east)
          to
          (m5b.south east)
          --
          cycle;
  \end{pgfonlayer}
\end{tikzpicture}%
\endgroup%
%
  }%
  \hfill%
  \adjustbox{valign=M, minipage=75mm}{%
    \captionsetup{skip=0pt}%
    \caption[%
              Example of sharing reduced nonterminals between nodes in a block
              DAG%
            ]%
            {%
              A block DAG to be covered using tree patterns~\cite{Ertl:1999}.
              %
              The nonterminal produced by a given match is given along the edge
              where the result is used.
              %
              Note that the \cVar*{a}~\gls{node} can be covered by two matches,
              both of which reduce the node to the same nonterminal.
              %
              Hence only one match is needed as the result can be shared by the
              two matches making use of that nonterminal%
            }
    \labelFigure{ertl-example}%
  }
\end{figure}
%
The \cVar*{reg}~\gls{node}, on the other hand, is reduced twice to the same
\gls{nonterminal} ($\mNT{Reg}$), and can thus be shared between the \glspl{rule}
that use this \gls{nonterminal} in the \glspl{pattern}.

As said earlier, however, this technique yields \gls{optimal.ps} \gls{pattern
  selection} only for certain \glspl{machine grammar}.
%
\citeauthor{Ertl:1999} therefore devised a checker, called \gls!{DBurg}, that
detects when the \gls{grammar} does not belong into this category and thus
cannot guarantee optimality.
%
The basic idea is to check whether every locally \gls{optimal.ps} decision is
also globally \gls{optimal.ps} by performing inductive proofs over the set of
all possible \glspl{block DAG}.
%
To do this efficiently, \citeauthor{Ertl:1999} implemented \gls{DBurg} using the
ideas behind \gls{Burg} (hence the name).


\paragraph{Combining DP and Edge Splitting}

\def\overlapCost{overlap-cost\xspace}
\def\cseCost{cse-cost\xspace}

\textcite{KoesGoldstein:2008} extended \citeauthor{Ertl:1999}'s ideas by
providing a heuristic that splits the \gls{block DAG} at points where \gls{node
  duplication} is estimated to have a detrimental effect on code quality.
%
Like \citeauthor{Ertl:1999}'s algorithm, \citeauthor{KoesGoldstein:2008}'s first
selects \glspl{pattern} optimally by performing a \gls{tree}-like, bottom-up
\gls{DP}~pass which ignores the fact that the input is a~\gls{DAG}.
%
Then, at points where multiple \glspl{pattern} overlap, two costs are
calculated:
%
\begin{inlinelist}[itemjoin={\ }, itemjoin*={\ and}]
  \item an \emph{\overlapCost}
  \item a \emph{\cseCost}
\end{inlinelist}.
%
The \overlapCost is an estimate of the cost of letting the \glspl{pattern}
overlap and thus incur duplication of operations in the final \gls{assembly
  code}.
%
The \cseCost is an estimate of the cost of splitting the \glspl{edge} at such
points.
%
If \cseCost is lower than \overlapCost, then the \gls{node} where overlapping
occurs is marked as \gls!{fixed.n}.
%
Once all such \glspl{node} have been processed, a second bottom-up \gls{DP}~pass
is performed on the \gls{block DAG}, but this time no \glspl{pattern} are
allowed to span across \gls{fixed.n} \glspl{node}, which can only be matched at
the \gls{root} of a \gls{pattern}.
%
Lastly, a top-down pass emits the \gls{assembly code}.

For evaluation purposes \citeauthor{KoesGoldstein:2008} compared their own
implementation, called \gls!{Noltis}, against an implementation based on
\glsdesc{IP} -- we will discuss such techniques later in this appendix -- and
found that \gls{Noltis} achieved \gls{optimal.ps} \gls{pattern selection} in
\SI{99.7}{\percent} of the test cases.
%
More details are given in \citeauthor{Koes:2009}'s doctoral
dissertation~\cite{Koes:2009}.
%
But like \citeauthor{Ertl:1999}'s design, \citeauthor{KoesGoldstein:2008}'s is
limited to \glspl{pattern tree} and thus cannot support more complex
\glspl{instruction} such as \gls{multi-output.ic} \glspl{instruction}.


\paragraph{Supporting Multi-output Instructions}

In most \gls{instruction selection} techniques based on \gls{DAG covering}, it
is assumed that the outputs of a \gls{pattern DAG} always occur at the
\gls{root}~\glspl{node}.
%
But in a design by \citeauthor{ArnoldCorporaal:1999}~\cite{ArnoldCorporaal:1999,
  ArnoldCorporaal:2001} (originally introduced in a technical report by
\textcite{Arnold:1999}), the \glspl{node} representing output can be marked
explicitly.
%
The advantage of this is that it allows the \glspl{pattern DAG} to be fully
decomposed into \glspl{tree} such that each output value receives its own
\gls{pattern tree}, which \citeauthor{ArnoldCorporaal:1999} call
\gls!{partial.p}[ \glspl{pattern}].
%
An example is given in \refFigure{arnold-example}.

\begin{figure}
  \mbox{}%
  \hfill%
  \subcaptionbox{Original pattern DAG\labelFigure{arnold-example-dag}}%
                [40mm]%
                {%
                  % Copyright (c) 2017-2018, Gabriel Hjort Blindell <ghb@kth.se>
%
% This work is licensed under a Creative Commons Attribution-NoDerivatives 4.0
% International License (see LICENSE file or visit
% <http://creativecommons.org/licenses/by-nc-nd/4.0/> for details).
%
\begingroup%
% Copyright (c) 2017-2018, Gabriel Hjort Blindell <ghb@kth.se>
%
% This work is licensed under a Creative Commons Attribution-NoDerivatives 4.0
% International License (see LICENSE file or visit
% <http://creativecommons.org/licenses/by-nc-nd/4.0/> for details).
%
\figureFont\figureFontSize%
\setlength{\nodeDist}{14pt}%
\setlength{\nodeSize}{12pt}%
\tikzset{%
  output node/.style={
    node,
    fill=black,
    text=white,
  },
}%
%
\begin{tikzpicture}
  \node [node] (x) {\nVar{x}};
  \node [node, right=of x] (y) {\strut\nVar{y}};
  \node [output node, above=of x] (add-output) {\nAdd};
  \node [node, above=of y] (add-not-output) {\nAdd};
  \node [node, right=of add-not-output] (zero) {\nVar{0}};
  \node [output node, above=of $(add-not-output) !.5! (zero)$] (eq) {\nEQ};
  \begin{scope}[data flow]
    \draw (add-not-output) -- (eq);
    \draw (zero) -- (eq);
    \draw (x) -- (add-not-output);
    \draw (y) -- (add-not-output);
    \draw (x) -- (add-output);
    \draw (y) -- (add-output);
  \end{scope}
\end{tikzpicture}%
\endgroup%
%
                }%
  \hfill%
  \subcaptionbox{Partial pattern trees\labelFigure{arnold-example-trees}}%
                {%
                  \adjustbox{valign=b}{%
                    % Copyright (c) 2017-2018, Gabriel Hjort Blindell <ghb@kth.se>
%
% This work is licensed under a Creative Commons Attribution-NoDerivatives 4.0
% International License (see LICENSE file or visit
% <http://creativecommons.org/licenses/by-nc-nd/4.0/> for details).
%
\begingroup%
% Copyright (c) 2017-2018, Gabriel Hjort Blindell <ghb@kth.se>
%
% This work is licensed under a Creative Commons Attribution-NoDerivatives 4.0
% International License (see LICENSE file or visit
% <http://creativecommons.org/licenses/by-nc-nd/4.0/> for details).
%
\figureFont\figureFontSize%
\setlength{\nodeDist}{14pt}%
\setlength{\nodeSize}{12pt}%
\tikzset{%
  output node/.style={
    node,
    fill=black,
    text=white,
  },
}%
%
\begin{tikzpicture}
  \node [node] (add) {\nAdd};
  \node [node, position=-135 degrees from add] (x) {\nVar{x}};
  \node [node, position=- 45 degrees from add] (y) {\strut\nVar{y}};
  \begin{scope}[data flow]
    \draw (x) -- (add);
    \draw (y) -- (add);
  \end{scope}
\end{tikzpicture}%
\endgroup%
%
                  }%
                  \hspace{1.5em}%
                  \adjustbox{valign=b}{%
                    % Copyright (c) 2017-2018, Gabriel Hjort Blindell <ghb@kth.se>
%
% This work is licensed under a Creative Commons Attribution-NoDerivatives 4.0
% International License (see LICENSE file or visit
% <http://creativecommons.org/licenses/by-nc-nd/4.0/> for details).
%
\begingroup%
% Copyright (c) 2017-2018, Gabriel Hjort Blindell <ghb@kth.se>
%
% This work is licensed under a Creative Commons Attribution-NoDerivatives 4.0
% International License (see LICENSE file or visit
% <http://creativecommons.org/licenses/by-nc-nd/4.0/> for details).
%
\figureFont\figureFontSize%
\setlength{\nodeDist}{14pt}%
\setlength{\nodeSize}{12pt}%
\tikzset{%
  output node/.style={
    node,
    fill=black,
    text=white,
  },
}%
%
\begin{tikzpicture}
  \node [node] (eq) {\nEQ};
  \node [node, position=-130 degrees from eq] (add) {\nAdd};
  \node [node, position=- 50 degrees from eq] (zero) {\nVar{0}};
  \node [node, position=-130 degrees from add] (x) {\nVar{x}};
  \node [node, position=- 50 degrees from add] (y) {\strut\nVar{y}};
  \begin{scope}[data flow]
    \draw (add) -- (eq);
    \draw (zero) -- (eq);
    \draw (x) -- (add);
    \draw (y) -- (add);
  \end{scope}
\end{tikzpicture}%
\endgroup%
%
                  }%
                }%
  \hfill%
  \mbox{}

  \caption[Example of converting a pattern DAG into partial tree patterns]%
          {%
            Example of converting a pattern DAG into partial pattern trees.
            %
            The pattern DAG represents an \instrCode*{add} instruction that also
            sets a status flag if the result is equal to~\num{0}.
            %
            The black nodes indicate the output nodes%
          }
  \labelFigure{arnold-example}
\end{figure}

The \gls{partial.p} \glspl{pattern} are then matched over the \gls{block DAG}
using an \mbox{$\mBigO(nm)$} algorithm.
%
After matching, another algorithm attempts to merge appropriate combinations of
partial \glspl{match} into \glspl{match} of the original \gls{pattern DAG}.
%
This is done in a straightforward manner by maintaining, for each \gls{match},
an array that maps the \glspl{node} in the \gls{pattern DAG} to the covered
\glspl{node} in the \gls{block DAG}.
%
Then, a check is made on whether two \gls{partial.p} \glspl{pattern} belong to
the same original \gls{pattern DAG} and have compatible mappings.
%
If a pair of \gls{pattern} \glspl{node} belong to different \gls{partial.p}
\glspl{pattern} but correspond to the same \gls{node} in the original
\gls{pattern DAG}, then both \gls{pattern} \glspl{node} must cover the same
\gls{node} in the \gls{block DAG}.
%
For \gls{pattern selection}, \citeauthor{ArnoldCorporaal:1999} applied a variant
of the \gls{DP}~scheme but combined it with a greedy heuristic in order to
enforce that each \gls{node} is covered exactly once.
%
Hence code quality is compromised.


\section{Modeling Instruction Selection as an M(W)IS Problem}
\labelSection{dc-mwis-based-approaches}

In the techniques discussed so far, the \gls{instruction selector} operates
directly on the \gls{block DAG} when performing \gls{pattern selection}.
%
The same applies for most designs based on \gls{tree covering}.
%
But another approach is to indirectly solve the \gls{pattern selection} problem
by first transforming it into an instance of some other problem for which there
already exist efficient solving methods.
%
When that problem has been solved, the answer can be translated back into a
solution for the original \gls{pattern selection} problem.

One such problem is the \gls!{MIS}[ problem], where the task is to select the
largest set of \glspl{node} from a \gls{graph} such that no pairs of selected
\glspl{node} have an \gls{edge} between them.
%
In the general case, finding such a solution is
NP-complete~\cite{GareyJohnson:1979}, and the \gls{pattern selection} problem is
transformed into an \gls{MIS} problem as follows.
%
From the \glspl{match set} found by \gls{pattern matching}, a corresponding
\gls!{conflict graph} -- or \gls!{interference graph}, as it is sometimes called
-- is formed.
%
Each \gls{node} in the \gls{conflict graph} represents a \gls{match}, and there
exists an \gls{edge} between two \glspl{node} if and only if the corresponding
\glspl{match} overlap.
%
An example of this is given in \refFigure{mis-example-2}.
%
\begin{figure}
  \centering%
  \mbox{}%
  \hfill%
  \subcaptionbox{%
                  Block DAG with matches%
                  \labelFigure{mis-example-2-dag}%
                }{%
                  % Copyright (c) 2017-2018, Gabriel Hjort Blindell <ghb@kth.se>
%
% This work is licensed under a Creative Commons Attribution-NoDerivatives 4.0
% International License (see LICENSE file or visit
% <http://creativecommons.org/licenses/by-nc-nd/4.0/> for details).
%
\begingroup%
\figureFont\figureFontSize%
\setlength{\nodeDist}{20pt}%
\pgfdeclarelayer{background1}%
\pgfdeclarelayer{background2}%
\pgfdeclarelayer{background3}%
\pgfsetlayers{background3,background2,background1,main}%
\begin{tikzpicture}[%
    value node/.style={%
      computation node,
    },
    outer match node/.style={%
      match node,
      draw=none,
      inner sep=0,
    },
  ]

  % Graph
  \node [computation node] (add) {\nAdd};
  \node [value node, position=-135 degrees from add] (a) {\nVar{a}};
  \node [value node, position=- 45 degrees from add] (b) {\nVar{b}};
  \node [computation node, position=135 degrees from add]
        (mul) {\nMul};
  \node [value node, position=45 degrees from add]
        (load) {\nLoad};
  \node [value node, position=-135 degrees from mul] (c) {\nVar{c}};
  \begin{scope}[data flow]
    \draw (a) -- (add);
    \draw (b) -- (add);
    \draw (add) -- (mul);
    \draw (add) -- (load);
    \draw (c) -- (mul);
  \end{scope}

  % Matches
  \begin{pgfonlayer}{background1}
    \node [match node, fill=shade2, fit=(a)] (m1) {};
    \node [match node, fill=shade2, fit=(b)] (m2) {};
    \node [match node, fill=shade2, fit=(c)] (m3) {};
    \node [match node, fill=shade2, fit=(add)] (m4) {};
    \node [match node, fill=shade2, fit=(mul)] (m5) {};
    \node [match node, fill=shade1, fit=(load)] (m7) {};
    % m6
    \begin{pgfonlayer}{background2}
      \node [outer match node, inner sep=-1pt, fit=(m5)] (m6a) {};
      \node [outer match node, inner sep=-1pt, fit=(m4)] (m6b) {};
      \draw [match line, bend left=45, fill=shade1]
            (m6a.south west)
            to
            (m6a.north west)
            to
            (m6a.north east)
            -- coordinate [pos=0] (m6)
            (m6b.north east)
            to
            (m6b.south east)
            to
            (m6b.south west)
            --
            cycle;
    \end{pgfonlayer}
    % m8
    \begin{pgfonlayer}{background3}
      \node [outer match node, inner sep=-1pt, fit=(m7)] (m8a) {};
      \node [outer match node, inner sep=1pt, fit=(m4)] (m8b) {};
      \draw [match line, bend left=45, fill=shade3]
            (m8a.north west)
            to
            (m8a.north east)
            to
            (m8a.south east)
            -- coordinate [pos=.25] (m8)
            (m8b.south east)
            to
            (m8b.south west)
            to
            (m8b.north west)
            --
            cycle;
    \end{pgfonlayer}
    % Fill in overdrawn edges of m8
    \begin{pgfonlayer}{background2}
      \draw [match line, bend left=45]
            (m8a.north west)
            to
            (m8a.north east)
            to
            (m8a.south east)
            --
            (m8b.south east)
            to
            (m8b.south west)
            to
            (m8b.north west)
            --
            cycle;
    \end{pgfonlayer}
  \end{pgfonlayer}

  % Match labels
  \node [match label, below=of m1] (m1l) {$m_1$};
  \node [match label, below=of m2] (m2l) {$m_2$};
  \node [match label, below=of m3] (m3l) {$m_3$};
  \node [match label, below=of m4] (m4l) {$m_4$};
  \node [match label, left=of m5] (m5l) {$\strut m_5$};
  \node [match label, position=45 degrees from m6, outer sep=-.5pt] (m6l)
        {$m_6$};
  \node [match label, right=of m7] (m7l) {$\strut m_7$};
  \node [match label, position=-45 degrees from m8] (m8l) {$m_8$};
  \foreach \i in {1, ..., 8} {
    \draw [match attachment line] (m\i) -- (m\i l);
  }
\end{tikzpicture}%
\endgroup%
%
                }%
  \hfill%
  \subcaptionbox{%
                  Interference graph%
                  \labelFigure{mis-example-2-int-graph}%
                }{%
                  % Copyright (c) 2017-2018, Gabriel Hjort Blindell <ghb@kth.se>
%
% This work is licensed under a Creative Commons Attribution-NoDerivatives 4.0
% International License (see LICENSE file or visit
% <http://creativecommons.org/licenses/by-nc-nd/4.0/> for details).
%
\begingroup%
\figureFont\figureFontSize%
\begin{tikzpicture}[%
    every node/.style={%
      node,
      inner sep=1pt,
    },
    interference/.style={%
      -,
      line width=\dataFlowLineWidth,
    },
  ]

  \node (m1) {$m_1$};
  \node [right=.5\nodeDist of m1] (m2) {$m_2$};
  \node [right=.5\nodeDist of m2] (m3) {$m_3$};
  \node [right=2\nodeDist of m3] (m4) {$m_4$};
  \node [position= 140 degrees from m4] (m5) {$m_5$};
  \node [position=-140 degrees from m4] (m6) {$m_6$};
  \node [position=  40 degrees from m4] (m7) {$m_7$};
  \node [position=- 40 degrees from m4] (m8) {$m_8$};

  \begin{scope}[interference]
    \draw (m6) -- (m4);
    \draw (m6) -- (m5);
    \draw (m8) -- (m4);
    \draw (m8) -- (m7);
    \draw (m6) -- (m8);
  \end{scope}
\end{tikzpicture}%
\endgroup%
%
                }%
  \hfill%
  \mbox{}

  \caption[Example of modeling instruction selection as a MIS problem]%
          {%
            Example of modeling instruction selection as a MIS problem.
            %
            Valid maximal independent sets of the interference graph are
            \mbox{$\mSet{m_1, \ldots, m_5, m_7}$}, \mbox{$\mSet{m_1, m_2, m_3,
                m_5, m_8}$}, and \mbox{$\mSet{m_1, m_2, m_3, m_6, m_7}$}, which
            correspond to valid covers of the block DAG%
          }
  \labelFigure{mis-example-2}
\end{figure}
%
By solving the \gls{MIS} problem for the \gls{conflict graph}, we obtain a
selection of \glspl{match} such that every \gls{node} in the \gls{block DAG} is
covered by exactly one \gls{match}.

But a solution to the \gls{MIS} problem does not necessarily yield an
\gls{optimal.ps} solution to the \gls{pattern selection} problem, as the former
does not incorporate costs.
%
We address this limitation by transforming the \gls{MIS} problem into a
\gls!{MWIS}[ problem], where the task is to find a solution to the \gls{MIS}
problem that maximizes (or minimizes) \mbox{$\sum_{p} \mWeight{p}$}, and assign
as weights the costs of the \glspl{pattern}.
%
We can get the solution with minimal total cost simply by negating the weights.
%
Note that although the \gls{MWIS}-based techniques discussed in this
dissertation have all been limited to \glspl{block DAG}, the approach can just
as well be applied in \gls{graph covering}, which will be introduced in
\refAppendix{graph-covering}.


\subsection{Applications}

In 2007, \textcite{ScharwaechterEtAl:2007} introduced what appears to be the
first \gls{instruction selection} technique to use the \gls{MWIS} approach for
selecting \glspl{pattern}.
%
But despite this novelty, the most cited contribution of their design is its
extensions to \glspl{machine grammar} to support \gls{multi-output.ic}
\glspl{instruction}.


\paragraph{Machine Grammars with Multiple Left-Hand Side Nonterminals}
\labelSection{dc-rules-multiple-productions}

\textcite{ScharwaechterEtAl:2007} appears to have pioneered the modeling of
\gls{instruction selection} as a \gls{MWIS} problem, although the main
contribution of their paper is the extension of \glspl{machine grammar} to
handle \gls{multi-output.ic} \glspl{instruction}.
%
The idea is to model such \glspl{instruction} using \gls!{complex.r}[
  \glspl{rule}], which each consists of multiple \glspl{production} -- one for
every result.
%
In this dissertation, such \glspl{production} and their \glspl{pattern} are
called \gls!{proxy.r}[ \glspl{rule}]%
%
\footnote{%
  In the original paper, they are called \gls!{split.r}[ \glspl{rule}].%
}
%
and \gls!{proxy.p}[ \glspl{pattern}], respectively, whereas \glspl{rule} with a
single \gls{production} and their \glspl{pattern} are called \gls!{simple.r}[
  \glspl{rule}] and \gls!{simple.p}[ \glspl{pattern}], respectively.
%
The \gls{rule} structure is also illustrated in
\refFigure{extended-machine-grammar-rule-anatomy-2}.

\begin{figure}
  \centering%
  \figureFont\figureFontSize%
  \newcommand{\simplePatternText}{%
    \trimbox{0 4pt 0 4pt}{%
      \begin{tabular}{@{}c@{}}
        simple\\[-1ex]
        pattern
      \end{tabular}%
    }%
  }
  \newcommand{\proxyPatternText}{%
    \trimbox{0 4pt 0 4pt}{%
      \begin{tabular}{@{}c@{}}
        proxy\\[-1ex]
        pattern
      \end{tabular}%
    }%
  }%
  \begin{displaymath}
    \underbrace{
      \mNT{A}
      \rightarrow
      \overbrace{
        \irCode{op} \ldots
      }^{\text{\simplePatternText}}
      \quad
      \text{cost}
      \quad
      \text{action}
    }_{\text{simple rule}}
    \qquad
    \underbrace{
      \langle \mNT{A}, \mNT{B}, \ldots \rangle
      \rightarrow
      \overbrace{
        \langle
          \:
          \overbrace{\irCode{op} \ldots}^{\text{\proxyPatternText}},
          \:
          \overbrace{\irCode{op} \ldots}^{\text{\proxyPatternText}},
          \:
          \ldots
          \:
        \rangle
      }^{\text{complex pattern}}
      \quad
      \text{cost}
      \quad
      \text{action}
    }_{\text{complex rule}}
  \end{displaymath}

  \vspace*{-\baselineskip}

  \caption{Anatomy of simple and complex rules in an extended machine grammar}
  \labelFigure{extended-machine-grammar-rule-anatomy-2}
\end{figure}

\Gls{pattern matching} is a two-step process.
%
First, the \glspl{match set} are found for the \gls{simple.p} and \gls{proxy.p}
\glspl{pattern}, using conventional \gls{tree}-based \gls{pattern matching}
techniques.
%
Second, the \glspl{match set} for the \gls{complex.p} \glspl{pattern} are found
by combining the \glspl{match} of \gls{proxy.p} \glspl{pattern} into
\glspl{match} of \gls{complex.p} \glspl{pattern} where appropriate.
%
The \gls{pattern selector} then checks whether it is worth applying a
\gls{complex.p} \gls{pattern} for covering a certain set of \glspl{node}, or if
they should be covered using the \gls{simple.p} \glspl{pattern} instead.
%
Since the intermediate results of \glspl{node} within \gls{complex.p}
\glspl{pattern} cannot be reused for other \glspl{pattern}, selecting a
\gls{complex.p} \gls{pattern} can incur an additional overhead cost as
\glspl{node} in the \gls{block DAG} may need to be covered using multiple
\glspl{pattern}.
%
Consequently, a \gls{complex.p} \gls{pattern} will only be selected if the cost
reduced by replacing a set of \gls{simple.p} \glspl{pattern} with this
\gls{pattern} is greater than the cost incurred by code duplication.

After these decisions have been taken, the next step is to perform \gls{pattern
  selection}.
%
For this, \citeauthor{ScharwaechterEtAl:2007} solve the corresponding \gls{MWIS}
problem in order to limit solutions to those of exact covering only.
%
The weights are calculated as the negated sum of the \gls{proxy.p} \gls{pattern}
costs, but the paper is ambiguous on how these costs are calculated.
%
Since the \gls{MWIS} problem is known to be NP-complete,
\citeauthor{ScharwaechterEtAl:2007} employed a greedy heuristic called
\gls!{Gwmin2} by \textcite{SakaiEtAl:2003}.
%
Lastly, \gls{proxy.p} \glspl{pattern} which have not been merged into
\gls{complex.p} \glspl{pattern} are replaced by corresponding \gls{simple.p}
\glspl{pattern} before \gls{assembly code} emission.

\citeauthor{ScharwaechterEtAl:2007} implemented a prototype called \gls!{CBurg}
as an extension of \gls{Olive} (see \refAppendix{tree-covering} on
\refPageOfSection{tc-olive}), and then ran some experiments by targeting a
\gls{MIPS}-like architecture.
%
In these experiments \gls{CBurg} generated \gls{assembly code} which improved
performance by almost \SI{25}{\percent}, and reduced code size by nearly
\SI{22}{\percent}, compared to \gls{assembly code} which was only allowed to
make use of \gls{single-output.ic} \glspl{instruction}.
%
Measurements of \gls{CBurg} also indicate that this technique exhibits
near-linear time complexity.
%
\textcite{AhnEtAl:2009} later broadened this work by including scheduling
dependency conflicts between \gls{complex.p} \glspl{pattern}, and incorporating
a feedback loop with the \gls{register allocator} to facilitate \gls{register
  allocation}.

A shortcoming of both designs by \citeauthor{ScharwaechterEtAl:2007} and
\citeauthor{AhnEtAl:2009} is that \gls{complex.r} \glspl{rule} can only consist
of disconnected \glspl{pattern tree} (hence there can be no sharing of
\glspl{node} between the \gls{proxy.p} \glspl{pattern}).
%
\textcite{YounEtAl:2011} address this problem in a 2011~paper -- which is a
revised and extended version of the original paper by
\citeauthor{ScharwaechterEtAl:2007} -- by introducing index subscripts for the
operand specification of the \gls{complex.r} \glspl{rule}.
%
However, the subscripts are restricted to the input \glspl{node} of the
\gls{pattern}, still hindering support for completely arbitrary \glspl{pattern
  DAG}.


\paragraph{Targeting Machines with Echo Instructions}

In 2004, \textcite{BriskEtAl:2004} introduced a technique to perform
\gls{instruction selection} for \glspl{target machine} with special \glspl!{echo
  instruction}, which are small markers that refer back to an earlier portion in
the \gls{assembly code} for re-execution.
%
This allows the \gls{assembly code} to be compressed by basically using the same
idea that is applied in the LZ77 algorithm~\cite{ZivLampel:1977}.\!%
%
\footnote{%
  The algorithm performs string compression by replacing recurring substrings
  that appear earlier in the string with pointers, allowing the original string
  to be reconstructed by ``copy-pasting.''%
}
%
Since \glspl{echo instruction} do not incur a branch or a procedure call, the
\gls{assembly code} can be reduced in size without sacrificing performance.
%
Consequently, unlike for traditional \glspl{target machine}, the \gls{pattern
  set} is not fixed in this case but must be determined as a precursor to
\gls{pattern matching}.
%
This is known as the \gls!{ISE}[ problem], which appears when generating code
for \glspl{ASIP} where the processor can be partially customized for executing a
particular \gls{program}.

The intuition behind this design is to use \glspl{echo instruction} where code
duplication is most prominent.
%
To find these cases in a given \gls{function}, \citeauthor{BriskEtAl:2004} first
enumerate all \glspl{subgraph} from the \glspl{block DAG}, and then match each
\gls{subgraph} over the \glspl{block DAG}.
%
\Gls{pattern matching} is done using \gls{VF2}, which is a generic \gls{subgraph
  isomorphism} algorithm (see \refChapter{existing-isel-techniques-and-reps} on
\refPageOfSection{ex-isel-rep-vf2-algorithm}).
%
Summing the sizes of the resulting \glspl{match set} gives a measure of code
duplication for each \gls{subgraph}, but this value will be an overestimation as
the \glspl{match set} may contain overlapping \glspl{match}.
%
\citeauthor{BriskEtAl:2004} addressed this by first solving the \gls{MIS}
problem on the \gls{conflict graph} for each \gls{match set}, and then adding up
the sizes of \emph{these} sets.
%
After selecting the most beneficial \gls{subgraph}, the covered \glspl{node} in
the \glspl{block DAG} are collapsed into single \glspl{node} to reflect the use
of \glspl{echo instruction}.
%
This process of matching and collapsing is then repeated until no new
\gls{subgraph} better than some user-defined value criterion can be found.
%
\citeauthor{BriskEtAl:2004} performed experiments on a prototype using a
selected set of benchmark applications, which showed code size reductions of
\SI{25}{\percent} to \SI{36}{\percent} on average.


\section[Modeling Instruction Selection as a Unate/Binate Covering Problem]%
        {Modeling Instruction Selection as a Unate/Binate Covering\\ Problem}
\labelSection{dc-unate-binate-covering}

Another approach to solving \gls{pattern selection} is to translate it to a
corresponding \glsshort!{unate covering} or \gls!{binate covering}[ problem].
%
The concepts behind the two are identical with the exception of one detail, and
both \glsshort{unate covering} and \gls{binate covering} can be used directly
for covering \glspl{graph} even though the designs discussed in this
dissertation have only been applied on \glspl{block DAG}.

Although \gls{binate covering}-based techniques actually appeared first, we will
begin with explaining \gls{unate covering}, as \gls{binate covering} is an
extension of \gls{unate covering}.


\paragraph{Unate Covering}

The idea of \gls{unate covering} is to create a Boolean matrix~$\mMatrix{M}$,
where each row represents a \gls{node} in the \gls{block DAG} and each column
represents a \gls{match} covering one or more \glspl{node} in the \gls{block
  DAG}.
%
If we denote $m_{ij}$ as row~$i$ and column~$j$ in~$\mMatrix{M}$, then
\mbox{$m_{ij} = 1$} indicates that \gls{node}~$i$ is covered by
\gls{pattern}~\mbox{$j$\hspace{-1pt}.}
%
Hence the \gls{pattern selection} problem is equivalent to finding a set of
columns in $\mMatrix{M}$ such that the sum for every row is exactly~\num{1}.
%
An example is given in \refFigure{unate-covering-example}.
%
\begin{figure}
  \subcaptionbox{%
                  Block DAG with matches%
                  \labelFigure{unate-covering-example-dag}%
                }%
                {%
                  ../existing-isel-techniques-and-reps/mis-example-dag.tex%
                }%
  \hfill%
  \subcaptionbox{%
                  Boolean matrix%
                  \labelFigure{unate-covering-example-matrix}%
                }%
                {%
                  \figureFont\figureFontSize%
                  \begin{tabular}{*{9}{c}}
                    \toprule
                        \tabhead node
                      & $\mathtabhead{m_1}$
                      & $\mathtabhead{m_2}$
                      & $\mathtabhead{m_3}$
                      & $\mathtabhead{m_4}$
                      & $\mathtabhead{m_5}$
                      & $\mathtabhead{m_6}$
                      & $\mathtabhead{m_7}$
                      & $\mathtabhead{m_8}$\\
                    \midrule
                        \tabhead \irVar{a}
                      & 1 & 0 & 0 & 0 & 0 & 0 & 0 & 0\\
                        \tabhead \irVar{b}
                      & 0 & 1 & 0 & 0 & 0 & 0 & 0 & 0\\
                        \tabhead \irVar{c}
                      & 0 & 0 & 1 & 0 & 0 & 0 & 0 & 0\\
                        \tabhead \irCode{\irAddText}
                      & 0 & 0 & 0 & 1 & 0 & 1 & 0 & 1\\
                        \tabhead \irCode{\irMulText}
                      & 0 & 0 & 0 & 0 & 1 & 1 & 0 & 0\\
                        \tabhead \irCode{\irLoadText}
                      & 0 & 0 & 0 & 0 & 0 & 0 & 1 & 1\\
                    \bottomrule

                  \end{tabular}%
                }

  \caption{Example of unate covering}
  \labelFigure{unate-covering-example}
\end{figure}
%
\Gls{unate covering} is an NP-complete problem, but as with the \gls{MIS} and
\gls{MWIS} problems there exist several efficient techniques for solving it
heuristically (see \cite{CordoneEtAl:2000, GoldbergEtAl:2006} for an overview).

\Gls{unate covering} alone, however, does not incorporate all necessary
\glspl{constraint} of \gls{pattern selection} when some \glspl{pattern} require
-- and prevent -- the selection of other \glspl{pattern} in order to yield
correct \gls{assembly code}.
%
Using \glspl{machine grammar} this can be enforced with the appropriate use of
\glspl{nonterminal}, but for \gls{unate covering} we have no means of expressing
this \gls{constraint}.
%
We therefore turn to \gls{binate covering}, where this is possible.


\paragraph{Binate Covering}

We first rewrite the Boolean matrix from the \gls{unate covering} problem into a
Boolean \gls{CNF} formula.
%
If \mbox{$\mVar{x}_i \in \mSet{0, 1}$} represents a \gls{variable} deciding
whether \gls{match}~$m_i$ is selected, then the Boolean matrix in
\refFigure{unate-covering-example-matrix} can be rewritten as
%
\begin{displaymath}
  \mVar{x}_1 \mAnd \mVar{x}_2 \mAnd \mVar{x}_3 \mAnd (\mVar{x}_4 \mOr \mVar{x}_6
  \mOr \mVar{x}_8) \mAnd (\mVar{x}_5 \mOr \mVar{x}_6) \mAnd (\mVar{x}_7 \mOr
  \mVar{x}_8).
\end{displaymath}

Now, the difference between \gls{unate covering} and \gls{binate covering} is
that all \glspl{variable} must be non-negated in the former, but may be negated
in the latter.
%
Hence \gls{binate covering} is equivalent to \gls{SAT}.


\subsection{Applications}

According to \citeauthor{LiaoEtAl:1995}~\cite{LiaoEtAl:1995, LiaoEtAl:1998} and
\textcite{CongEtAl:2004}, the pioneering use of \gls{binate covering} to solve
\gls{DAG covering} was done by \textcite{Rudell:1989} in 1989 as a part of a
\gls!{VLSI} synthesis design.
%
\citeauthor{LiaoEtAl:1995}~\cite{LiaoEtAl:1995, LiaoEtAl:1998} later adapted it
to \gls{instruction selection} in a method that optimizes code size for
one-\gls{register} \glspl{target machine}.
%
To prune the search space, \citeauthor{LiaoEtAl:1995} perform \gls{pattern
  selection} in two iterations.
%
In the first iteration, \glspl{pattern} are selected such that the \gls{block
  DAG} is covered but the costs of necessary data transfers are ignored.
%
After this step the \glspl{node} covered by the same \gls{pattern} are collapsed
into single \glspl{node}, and a second \gls{binate covering} problem is
constructed to minimize the costs of data transfers.
%
Although these two problems can be solved simultaneously,
\citeauthor{LiaoEtAl:1995} chose not to do so as the number of necessary
implication clauses would become very large.
%
Recently, \textcite{CongEtAl:2004} also applied \gls{binate covering} as part of
generating application-specific \glspl{instruction} for configurable processor
architectures.

\Gls{unate covering} was applied by \textcite{ClarkEtAl:2006} in generating
\gls{assembly code} for acyclic computation accelerators, which can be partially
customized in order to increase performance of the currently executed
\gls{function}.
%
Described in a paper from~2006, the \glspl{target machine} are presumably
homogeneous enough that implication clauses are unnecessary.
%
The work by \citeauthor{ClarkEtAl:2006} was later expanded by
\textcite{HormatiEtAl:2007} to reduce the number of interconnects as well as
data-centered latencies in accelerator designs.

\citeauthor{MartinEtAl:2009}~\cite{MartinEtAl:2009, MartinEtAl:2012} also
applied \gls{unate covering} to solve a similar problem concerning
reconfigurable processor extensions.
%
However, they combined \gls{instruction selection} with \gls{instruction
  scheduling} and solved both in tandem using \glsdesc{CP} -- we will discuss
this approach later in this appendix -- which they also applied for solving the
\gls{pattern matching} problem.
%
Unlike in the cases of \citeauthor{ClarkEtAl:2006} and
\citeauthor{HormatiEtAl:2007}, who solved their \gls{unate covering} problems
using heuristics, the \gls{assembly code} generated by
\citeauthor{MartinEtAl:2009} is potentially optimal.


\section{Modeling Instruction Selection Using IP}
\labelSection{dc-ip-based-approaches}

It is well known that performing \gls{instruction selection}, \gls{instruction
  scheduling}, or \gls{register allocation} in isolation will typically always
yield suboptimal \gls{assembly code}.
%
But since each subproblem is already NP-complete on its own, attaining
\gls!{integrated.cg}[ \gls{code generation}] -- where all these problems are
solved simultaneously -- is an even more difficult problem.

These challenges notwithstanding, \textcite{WilsonEtAl:1994} introduced in 1994
what appears to be the first design that could be said to yield truly optimal
\gls{assembly code}.
%
\citeauthor{WilsonEtAl:1994} accomplished this by using \gls!{IP}, which is a
method for solving combinatorial optimization problems (sometimes \gls{IP} is
also called \glsdesc!{ILP}).
%
In \gls{IP}, a problem is expressed using sets of integer \glspl{variable} and
linear equations, and a \gls{solution} to an \gls{IP}~model is an assignment to
all \glspl{variable} such that all equations are fulfilled (see
\refDefinition{ip} on \refPageOfDefinition{ip} for a formal definition).
%
In general, solving an \gls{IP}~model is NP-complete, but extensive research in
this field has made many problem instances tractable.
%
For a comprehensive overview of IP, see in~\cite{Wolsey:1998}.


\paragraph{Modeling Pattern Selection Using Linear Inequality}

In their seminal paper, \citeauthor{WilsonEtAl:1994} describe that the
\gls{pattern selection} problem can be expressed as the following linear
inequality:
%
\begin{displaymath}
  \forall n \in N : \sum_{\mathclap{m \,\in\, M_n}} \mVar{x}_m \leq 1.
\end{displaymath}
%
This reads: for every \gls{node}~$n$ in the \gls{block
  DAG}~\mbox{$\mTuple{N\hspace{-1pt}, E}$,} at most one \gls{match}~$m$ from the
\gls{match set} involving~$n$ (represented by $M_n$) may be selected.\!%
%
\footnote{%
  The more common \gls{constraint} is that \emph{exactly one} \gls{match} must
  be selected, but in the design by \citeauthor{WilsonEtAl:1994} \glspl{node}
  are allowed to become inactive and thus need not be covered.%
}
%
The decision is represented by a $\mVar{x}_m \in \mSet{0, 1}$~\gls{variable}.

Similar linear equations can be formulated for modeling \gls{instruction
  scheduling} and \gls{register allocation} -- which
\citeauthor{WilsonEtAl:1994} also included in their model -- but these are out
of scope for this dissertation.
%
In fact, any \gls{constraint} that can be formulated in this way can be added to
an existing \gls{IP}~model, making this approach a suitable \gls{code
  generation} method for targeting irregular architectures.
%
Furthermore, this is the first design we have seen that could potentially
support \gls{interdependent.ic} \glspl{instruction} (although this was not the
main focus of \citeauthor{WilsonEtAl:1994}).

Solving this monolithic \gls{IP}~model, however, typically requires considerably
more time compared to the previously discussed techniques of \gls{instruction
  selection}.
%
But the trade-off for longer compilation time is better code quality;
\citeauthor{WilsonEtAl:1994} reported that experiments showed that the generated
\gls{assembly code} was of comparable code quality to that of manually optimized
\gls{assembly code}.
%
In theory, optimal \gls{assembly code} can be generated, although this is in
practice only feasible for small enough \glspl{function}.
%
Another much-valued feature is the ability to extend the model with additional
\glspl{constraint} in order to support complicated \glspl{target machine}, which
cannot be properly handled by the conventional designs as that typically
violates assumptions made by the underlying heuristics.


\subsection{Approaching Linear Solving Time with Horn Clauses}

\newcommand{\mCmd}[1]{\textsf{#1}}

Although \gls{IP}~models are NP-complete to solve in general, it was discovered
that for a certain class of problem instances -- namely those based on
\glspl{Horn clause} -- an optimal solution can be found in linear
time~\cite{Hooker:1988}.
%
A \gls!{Horn clause} is a disjunctive Boolean formula which contains at most one
non-negated term.
%
This can also be phrased as a logical statement that has at most one conclusion.
%
For example, the statement
%
\begin{displaymath}
  \mCmd{if}~\mVar{x}_1~\mCmd{and}~\mVar{x}_2~\mCmd{then}~\mVar{x}_3
\end{displaymath}
%
can be expressed as \mbox{$\mNot\mVar{x}_1 \mOr \mNot\mVar{x}_2 \mOr
  \mVar{x}_3$}, which is a \gls{Horn clause}, as only $\mVar{x}_3$ is not
negated.
%
This can then easily be rewritten into the linear inequality
%
\begin{displaymath}
(1 - \mVar{x}_1) + (1 - \mVar{x}_2) + \mVar{x}_3 \ge 1.
\end{displaymath}
%
Moreover, statements that do not yield \glspl{Horn clause} in their current form
can often be rewritten so that they do.
%
For example,
%
\begin{displaymath}
  \mCmd{if}~\mVar{x}_1~\mCmd{then}~\mVar{x}_2~\mCmd{and}~\mVar{x}_3
\end{displaymath}
%
can be expressed as \mbox{$\mNot\mVar{a} \mOr \mVar{b} \mOr \mVar{c}$} and is
thus not a \gls{Horn clause} because it has more than one non-negated term.
%
But by rewriting it into
%
\begin{displaymath}
  \begin{array}{c}
    \mCmd{if}~\mVar{x}_1~\mCmd{then}~\mVar{x}_2\\
    \mCmd{if}~\mVar{x}_1~\mCmd{then}~\mVar{x}_3
  \end{array}
\end{displaymath}
%
the statement can now be expressed as \mbox{$\mNot\mVar{x}_1 \mOr \mVar{x}_2$}
and \mbox{$\mNot\mVar{x}_1 \mOr \mVar{x}_3$}, which are two valid \glspl{Horn
  clause}.

\textcite{Gebotys:1997} exploited this property in 1997 by developing an
\gls{IP}~model for \gls{TMS320C2x} -- a common \gls{DSP} at the time -- where
many of the \glspl{constraint} imposed by the target architecture,
\gls{instruction selection}, and \gls{register allocation}, and a part of the
\gls{instruction scheduling} problem, are expressed as \glspl{Horn clause}.
%
Using only \glspl{Horn clause} may require a larger number of \glspl{constraint}
than are otherwise needed, but \citeauthor{Gebotys:1997} claims that the number
is still manageable.
%
When compared against a then-contemporary industrial \gls{DSP} \gls{compiler},
\citeauthor{Gebotys:1997} demonstrated that an implementation based on \gls{IP}
yielded a performance improvement mean of \SI{44}{\percent} for a select set of
functions, while attaining reasonable compilation times.
%
However, the solving time increased by orders of magnitude when
\citeauthor{Gebotys:1997} augmented the \gls{IP}~model with the complete set of
\glspl{constraint} for \gls{instruction scheduling}, which cannot be expressed
entirely as \glspl{Horn clause}.


\subsection{IP-Based Designs with Multi-output Instruction Support}

\citeauthor{LeupersMarwedel:1996}~\cite{LeupersMarwedel:1995,
  LeupersMarwedel:1996} expanded the work of \citeauthor{WilsonEtAl:1994} --
whose design is restricted to \glspl{pattern tree} -- by developing an
\gls{IP}-based \gls{instruction selector} which also supports
\gls{multi-output.ic} \glspl{instruction}.
%
In a paper from~:1996, \citeauthor{LeupersMarwedel:1996} describe a scheme where
the \glspl{pattern DAG} of \gls{multi-output.ic} \glspl{instruction} --
\citeauthor{LeupersMarwedel:1996} refer to these as \gls{complex.p}
\glspl{pattern} -- are first decomposed into multiple \glspl{pattern tree}
according to their \glspl{RT}.
%
\glspl{RT}~are akin to \citeauthor{Fraser:1979}'s \glspl{RTL}~\cite{Fraser:1979}
(see \refAppendix{macro-expansion} on
\refPageOfSection{me-register-transfer-lists}), and essentially mean that each
observable effect gets its own \gls{pattern tree}.
%
Each individual \gls{RT} may in turn correspond to one or more
\glspl{instruction}, but unlike in \citeauthor{Fraser:1979}'s design this is not
strictly required.

Assuming the \gls{block DAG} has already been \glsshort{undagging}[ged], each
\gls{expression tree} is first optimally covered using \gls{IBurg}.
%
The \glspl{RT} are expressed as \glspl{rule} in an \gls{machine grammar} that
has been automatically generated from a \gls{machine description} written in
\glsunset{MIMOLA}\gls{MIMOLA}\glsreset{MIMOLA} (we will come back to this in
\refSection{dc-modeling-entire-target-machines}).
%
Once \glspl{RT} have been selected, the \gls{expression tree} is reduced to a
\gls{tree} of \glspl{super node}, where each \gls{super node} represents a set
of \glspl{node} covered by some \gls{RT} that have been collapsed into a single
\gls{node}.
%
Since \gls{multi-output.ic} and \gls{disjoint-output.ic} \glspl{instruction}
implement more than one~\gls{RT}, the goal is now to cover the \gls{super node}
\gls{graph} using the \glspl{pattern} which are formed when the
\glspl{instruction} are modeled as~\glspl{RT}.
%
\citeauthor{LeupersMarwedel:1996} addressed this problem by applying a modified
version of the \gls{IP}~model by \citeauthor{WilsonEtAl:1994}.

But because the step of selecting \glspl{RT} to cover the \gls{expression tree}
is separate from the step which implements them with \glspl{instruction}, the
generated \gls{assembly code} is not necessarily optimal for the whole
\gls{expression tree}.
%
To achieve this property, the covering of \glspl{RT} and selection of
\glspl{instruction} must be done in tandem.


\subsection{IP-Based Designs with Disjoint-output Instruction Support}

\textcite{Leupers:2000:SIMD} later made a more direct extension of the
\gls{IP}~model by \citeauthor{WilsonEtAl:1994} in order to support \gls{SIMD.i}
\glspl{instruction}, which belong to the class of \gls{disjoint-output.ic}
\glspl{instruction}.
%
Described in a paper from~2000, \citeauthor{Leupers:2000:SIMD}'s design assumes
every \gls{SIMD.i} \gls{instruction} performs two operations, each of which
takes a disjoint set of input operands.
%
This is collectively called a \gls!{SIMD pair}, and
\citeauthor{Leupers:2000:SIMD} then extended the \gls{IP}~model with linear
equations for combining \glspl{SIMD pair} into \gls{SIMD.i} \glspl{instruction}
and defined the objective function so as to maximize the use of \gls{SIMD.i}
\glspl{instruction}.

In the paper, \citeauthor{Leupers:2000:SIMD} reports experiments where the use
of \gls{SIMD.i} \glspl{instruction} reduced code size by up to \SI{75}{\percent}
for the selected test cases and \glspl{target machine}.
%
But since this technique assumes that each individual \gls{operation} of the
\gls{SIMD.i} \glspl{instruction} is expressed as a single \gls{node} in the
\gls{block DAG}, it is unclear whether the method can be extended to more
complex \gls{SIMD.i} \glspl{instruction} and whether it scales to larger
\glspl{function}.
%
\textcite{TanakaEtAl:2003} later expanded \citeauthor{Leupers:2000:SIMD}'s work
for selecting \gls{SIMD.i} \glspl{instruction} while also taking \gls{data
  copying} into account by introducing auxiliary transfer \glspl{node} and
transfer \glspl{pattern} into the \gls{block DAG}.


\subsection{Modeling the Pattern Matching Problem with IP}

In 2006, \textcite{BednarskiKessler:2006} developed an \gls{integrated.cg}
\gls{code generation} design where both \gls{pattern matching} and \gls{pattern
  selection} are solved using \glsdesc{IP}.
%
The scheme -- which later was applied by \textcite{ErikssonEtAl:2008}, and is
also described in an article by \textcite{ErikssonKessler:2012} -- is an
extension of their earlier work where \gls{instruction selection} had previously
more or less been ignored (see~\cite{KesslerBednarski:2001,
  KesslerBednarski:2002}).

In broad outline, the \gls{IP}~model assumes that a sufficient number of
\glspl{match} has been generated for a given \gls{block DAG}~$G$.
%
This is done using a \gls{pattern matching} heuristic that computes an upper
bound.
%
For each \gls{match}~\mbox{$m$\hspace{-.8pt},} the \gls{IP}~model contains
integer \glspl{variable} that:
%
\begin{itemize}
  \item map a \gls{pattern} \gls{node} in $m$ to a \gls{node} in $G$;
  \item map a \gls{pattern} \gls{edge} in $m$ to an \gls{edge} in $G$; and
  \item decide whether $m$ is used in the solution.
    %
    Remember that we may have an excess of \glspl{match}, so they cannot all be
    selected.
\end{itemize}
%
Hence, in addition to the typical linear equations we have seen previously for
enforcing coverage, this \gls{IP}~model also includes equations to ensure that
the selected \glspl{match} are valid \glspl{match}.

Implemented in a framework called \gls!{Optimist},
\citeauthor{BednarskiKessler:2006} compared their \gls{IP}~model against another
\gls{integrated.cg} \gls{code generation} design based on \glsdesc{DP}.
%
This \gls{DP} algorithm, however, which was developed by the same authors (see
\cite{KesslerBednarski:2001}), has nothing to do with the conventional
\gls{DP}~algorithm by \textcite{AhoEtAl:1989}).
%
\citeauthor{BednarskiKessler:2006} found that \gls{Optimist} substantially
reduced compilation time while retaining code quality.
%
However, for several test cases -- the largest \gls{block DAG} containing only
\num{33}~\glspl{node} -- \gls{Optimist} failed to generate any \gls{assembly
  code} whatsoever within the set time limit.
%
One reasonable cause could be that the \gls{IP}~model also attempts to solve
\gls{pattern matching} -- a problem which we have seen can be solved externally
-- and thus further exacerbates an already computationally difficult problem.


\section{Modeling Instruction Selection Using CP}
\labelSection{dc-cp-based-approaches}

Although \glsdesc{IP} allows auxiliary \glspl{constraint} to be included into
the \gls{IP}~model, they may be cumbersome to express as linear equations.
%
This issue can be alleviated by using \gls!{CP}, which is another method for
solving combinatorial optimization problems but has more flexible modeling
capabilities compared to \gls{IP}.
%
For a brief introduction to \gls{CP}, see \refChapter{constraint-programming}.


\paragraph{First Application}

In 1990, \textcite{BashfordLeupers:1999} pioneered the use of \gls{CP} in
\gls{code generation} by developing a \gls{constraint model} for
\gls{integrated.cg} \gls{code generation} that targets \glspl{DSP} with highly
irregular architectures (the work is also discussed in~\cite{Leupers:2000:SIMD,
  LeupersBashford:2000}).
%
Like \citeauthor{LeupersMarwedel:1996}'s \gls{IP}-based design,
\citeauthor{BashfordLeupers:1999}'s first breaks down the \gls{instruction set}
of the \gls{target machine} into a set of \glspl!{RT} which are used to cover
individual \glspl{node} in the \gls{block DAG}.
%
As each \gls{RT} concerns specific \glspl{register} on the \gls{target machine},
the covering problem essentially also incorporates \gls{register allocation}.
%
The goal is then to minimize the cost of covering by combining multiple
\glspl{RT} that can be executed in parallel as part of some \gls{instruction}.

For each \gls{node} in the \gls{block DAG} a \gls{FRT} is introduced, which
basically embodies all \glspl{RT} that match a particular \gls{node} and is
formally defined as the following tuple:
%
\begin{displaymath}
  \mTuple*{
    \mathit{op},
    \mVar{d},
    [\mVar{u}_1, \ldots, \mVar{u}_n],
    \mVar{f},
    \mVar{c},
    \mVar{t},
    \mathit{CS}
  }.
\end{displaymath}
%
$\mathit{op}$ is the operation of the \gls{node}.
%
$\mVar{d}$ and \mbox{$\mVar{u}_1, \ldots, \mVar{u}_n$} are \glspl{variable}
representing the \glspl{storage location} of the result and the respective
inputs to the operation.
%
These are typically the \glspl{register} that can be used for the operation, but
also include \gls!{virtual.sl}[ \glspl{storage location}] which convey that the
value is produced as an intermediate result in a chain of operations (for
example, the multiplication term in a multiply-accumulate instruction is such a
result).
%
Then, for every pair of \glspl{operation} that are \gls{adjacent.n} in the
\gls{block DAG}, a set of \glspl{constraint} is added to ensure that there
exists a valid data transfer between the \glspl{storage location} of $\mVar{d}$
and $\mVar{u}_i$ if these are assigned to different \glspl{register}.
%
In addition, if either of $\mVar{d}$ and $\mVar{u}_i$ resides in a
\gls{virtual.sl} \gls{storage location}, then both must be identical.
%
$\mVar{f}$,~$\mVar{c}$, and $\mVar{t}$ are all \glspl{variable} which
collectively represent the \gls!{ERI}.
%
The \gls{ERI} specifies at which functional unit the operation will be executed,
at what cost in terms of number of execution cycles, and by which
\gls{instruction} type.
%
A combination of a functional unit and an \gls{instruction} type is later mapped
to a particular \gls{instruction}.
%
Multiple \glspl{RT} can be combined into the same \gls{instruction} when the
destination of the result is a \gls{virtual.sl} \gls{storage location} by
setting \mbox{$\mVar{c} = 0$} and letting the last \gls{node} in the operation
chain account for the required number of execution cycles.
%
The last entity, $\mathit{CS}$, is the set of \glspl{constraint} for defining
the range of values for the \glspl{variable} and the dependencies between
$\mVar{d}$ and~$\mVar{u}_i$, as well as other auxiliary \glspl{constraint} that
may be required for the \gls{target machine}.
%
For example, if the set of \glspl{RT} matching a \gls{node} consists of
%
\mbox{$%
  \mSet{
    \cVar{r}[c] = \cVar{r}[a] + \cVar{r}[b],
    \cVar{r}[a] = \cVar{r}[c] + \cVar{r}[b]
  }
$}, then the corresponding \gls!{FRT} becomes
%
\begin{displaymath}
  \mTuple*{
    +,
    \mVar{d},
    [\mVar{u}_1, \mVar{u}_2],
    \mVar{f},
    \mVar{c},
    \mVar{t},
    \mSet*{
      \mVar{d} \in \mSet{\cVar{r}[c], \cVar{r}[a]},
      \mVar{u}_1 \in \mSet{\cVar{r}[a], \cVar{r}[c]},
      \mVar{u}_2 = \cVar{r}[b],
      \mVar{d} = \cVar{r}[c] \mImp \mVar{u}_1 = \cVar{r}[a]
    }
  }\!.
\end{displaymath}
%
For brevity, we omit several details such as the \glspl{constraint} concerning
the~\gls{ERI}.

This \gls{constraint model} is then solved to optimality using a \gls{constraint
  solver}.
%
But since \gls{optimal.ps} \gls{cover}[ing] using \glspl{FRT} is NP-complete,
\citeauthor{BashfordLeupers:1999} applied heuristics to curb the complexity by
splitting the \gls{block DAG} into smaller pieces along \glspl{edge} where
intermediate results are shared.
%
Once split, \gls{instruction selection} is then performed on each
\gls{expression tree} in isolation.


\subsection{Taking Advantage of Global Constraints}

So far we have discussed several techniques that apply \glsdesc{CP} for solving
the problems of \gls{pattern matching} and \gls{pattern selection} -- namely
those by \citeauthor{BashfordLeupers:1999} and \citeauthor{MartinEtAl:2009}.
%
Recently, \textcite{Beg:2013} introduced another \gls{constraint model} for
\gls{instruction selection} as well as new methods for improving solving.
%
For example, in order to reduce the search space, \citeauthor{Beg:2013} applied
conventional \gls{DP}-based techniques to compute an upper bound on the cost.
%
However, the \gls{constraint model} mainly deals with the problem of
\gls{pattern matching} rather than \gls{pattern selection}.
%
Moreover, \citeauthor{Beg:2013} noticed only a negligible improvement (less than
\SI{1}{\percent}) in code quality compared to \gls{LLVM}, mainly because the
\glspl{target machine} (\gls{MIPS} and \gls{ARM}) were simple enough that greedy
heuristics generate near-optimal \gls{assembly code}.
%
In addition, the \glspl{block DAG} of the benchmark \glspl{function} were fairly
\gls{tree}-shaped~\cite{VanBeek:2014}, for which \gls{optimal.ps} code can be
generated in linear time.
%
In any case, none of these designs take advantage of a key feature of
\glsdesc{CP}, which is the use of \gls{global.c} \glspl{constraint}.
%
A \gls!{global.c}[ \gls{constraint}] captures relations among multiple
\glspl{variable} and results in more \gls{search space} pruning than if it had
been expressed using a decomposition of \glspl{constraint}.

Hence, when \textcite{FlochEtAl:2010} in 2010 adapted the \gls{constraint model}
by \citeauthor{MartinEtAl:2009} to support processors with reconfigurable cell
fabric, they replaced the method of \gls{pattern selection} with
\glspl{constraint} that are radically different from those incurred by
\gls{unate covering}.
%
In addition, unlike in the case of \citeauthor{BashfordLeupers:1999}, the design
by \citeauthor{FlochEtAl:2010} applies the more direct form of \gls{pattern
  matching} instead of first breaking down the \glspl{pattern} into \glspl{RT}
and then selecting \glspl{instruction} that combine as many \glspl{RT} as
possible.


\paragraph{Modeling Pattern Selection Using Global Cardinality Constraint}

As described in their 2010~paper, \citeauthor{FlochEtAl:2010} use the
\gls{global cardinality constraint} to enforce the requirement that every
\gls{node} in the \gls{block DAG} must be covered by exactly one
\gls{match}.\!%
%
\footnote{%
  It is also possible to enforce \gls{pattern selection} through a \gls{global
    set covering constraint} developed by \textcite{MouthuyEtAl:2007}, but
  no implementation is known to do so.%
}
%
The \gls{constraint}, referred to as $\mGCC$, constrains the number of
\glspl{variable} assigned a particular value (which may also be a
\gls{variable}).
%
Given a set \mbox{$v_1, \ldots, v_k$} of values and two sets \mbox{$\mVar{x}_1,
  \ldots, \mVar{x}_n$} and \mbox{$\mVar{c}_1, \ldots, \mVar{c}_k$} of
\glspl{variable}, the \gls{constraint} holds if, for each \mbox{$i = 1, \ldots,
  k$\hspace{-.8pt},} exactly $\mVar{c}_i$ \glspl{variable} in the set
\mbox{$\mVar{x}_1, \ldots, \mVar{x}_n$} are assigned value~$v_i$ (see also
\refDefinition{gcc} on \refPageOfDefinition{gcc}).
%
For example, \mbox{$\mGCC(\mTuple{5, \mVar{c}_1 = 0} \hspace{-1pt}, \mTuple{3,
    \mVar{c}_2 = 1} \hspace{-1pt}, \mVar{x}_1 = 2, \mVar{x}_2 =3)$} holds
because no $\mVar{x}$~\gls{variable} is assigned value~\num{5} and exactly one
$\mVar{x}$~\gls{variable} is assigned value~\num{3}.
%
Similarly, \mbox{$\mGCC(\mTuple{3, \mVar{c}_1 \in \mSet{0, 2}} \hspace{-1pt},
  \mVar{x}_1 = 2, \mVar{x}_2 = 3)$} does not holds because either none or both
$\mVar{x}$~\glspl{variable} must be assigned value~\num{3}.

To model \gls{pattern selection} using $\mGCC$, two new sets of \glspl{variable}
are needed.
%
Assume that $N$ denotes the set of \glspl{node} to be covered, $M$ denotes the
\gls{match set}, and $\mCovers(m)$ denotes the set of \glspl{node} covered by
\gls{match}~\mbox{$m$\hspace{-.8pt}.}
%
Then, \gls{variable} \mbox{$\mVar{match}_n \in \mSetBuilder{m}{m \in M, n \in
    \mCovers(m)}$} decides which \gls{match} covers
\gls{node}~\mbox{$n$\hspace{-.8pt},} and \gls{variable} \mbox{$\mVar{count}_m
  \in \mSet{0, |\mCovers(m)|}$} decides how many \glspl{node} are covered by
\gls{match}~\mbox{$m$\hspace{-.8pt}.}
%
Hence each match covers either no \glspl{node} or all \glspl{node} in its
\gls{pattern}.
%
With these \glspl{variable}, \gls{pattern selection} can be modeled as
%
\begin{displaymath}
  \mGCC(
    \cup_{m \,\in\, M} \mTuple{m, \mVar{count}_m} \hspace{-1pt},
    \cup_{n \,\in\, N} \, \mVar{match}_n
  ),
\end{displaymath}
%
which offers stronger \gls{propagation} than the corresponding linear inequality
constraint and thus reduces solving time~\cite{FlochEtAl:2010}.


\paragraph{Accommodating VLIW Architectures}

The \gls{constraint model} by \citeauthor{FlochEtAl:2010} was also further
extended by \citeauthor{ArslanKuchcinski:2013}~\cite{ArslanKuchcinski:2013,
  ArslanKuchcinski:2014, Arslan:2016} to accommodate \gls{VLIW} architectures
and \gls{disjoint-output.ic} \glspl{instruction}.
%
First, every \gls{disjoint-output.ic} \glspl{instruction} is split into multiple
\glspl!{subinstruction}, each modeled by a disjoint \gls{pattern} which is
mapped onto the \gls{block DAG} using a generic \gls{subgraph isomorphism}
algorithm.
%
\Gls{pattern selection} is then modeled as an instance of the \gls{constraint
  model} with the additional \glspl{constraint} to schedule the
\glspl{subinstruction} such that they can be replaced by the original
\gls{disjoint-output.ic} \gls{instruction}.
%
Consequently, unlike previous techniques that recombine \gls{partial.p}
\glspl{match} into \gls{complex.p} \glspl{match} prior to \gls{pattern
  selection} (see for example \textcite{ScharwaechterEtAl:2007},
\textcite{AhnEtAl:2009}, \citeauthor{ArnoldCorporaal:1999}~\cite{Arnold:1999,
  ArnoldCorporaal:1999, ArnoldCorporaal:2001}),
\citeauthor{ArslanKuchcinski:2013} instead solve these two problems in tandem.
%
Their design is also capable of accepting multiple, disconnected \glspl{block
  DAG} as a single input.


\paragraph{Limitations}

An inherent limitation to the \glspl{constraint model} applied by
\citeauthor{MartinEtAl:2009}, \citeauthor{FlochEtAl:2010}, and
\citeauthor{ArslanKuchcinski:2013} is that they do not model the necessary data
transfers between different \glspl{register class}.
%
This in turn means that the cost model is only accurate for \glspl{target
  machine} equipped with a homogeneous \gls{register} architecture, which could
compromise code quality for more complicated \glspl{target machine}.


\section{Other DAG-Based Approaches}
\labelSection{dc-other-approaches}

\subsection{More Genetic Algorithms}

Seemingly independently from the earlier work by \textcite{ShuEtAl:1996}
(discussed in \refAppendix{tree-covering} on
\refPageOfSection{tc-genetic-algorithms}),
\citeauthor{LorenzEtAl:2001}~\cite{LorenzEtAl:2001, LorenzMarwedel:2004}
introduced in 2001 another technique where \glsdesc{GA}s are applied to
\gls{code generation}.
%
But unlike the design by \citeauthor{ShuEtAl:1996}, the one by
\citeauthor{LorenzEtAl:2001} takes \glspl{block DAG} instead of \glspl{tree} as
input and also incorporates \gls{instruction scheduling} and \gls{register
  allocation}.
%
\citeauthor{LorenzEtAl:2001} recognized that contemporary \glspl{compiler}
struggled with generating efficient \gls{assembly code} for \glspl{DSP} equipped
with very few \glspl{register} and typically always spill the results of common
subexpressions to memory and reload them when needed.
%
Compared to optimal \gls{assembly code}, this may incur more memory accesses
than needed.

The design by \citeauthor{LorenzEtAl:2001} is basically an iterative process.
%
First, the operations within a \gls{block} are scheduled using \gls!{list
  scheduling}, which is a traditional method of scheduling (see for
example~\cite{RauFisher:1993}).
%
For every scheduled operation, a \gls{gene} is formulated to encode all the
possible decisions to take in order to solve the problems of \gls{instruction
  selection} and \gls{register allocation}.
%
These decisions are then taken over multiple steps using standard
\gls{GA}~operations, where the values are selected probabilistically.
%
In each step the \gls{gene} is mutated and crossed over in order to produce new,
hopefully better \glspl{gene}, and a \gls{fitness function} is applied to
evaluate each \gls{gene} in terms of expected execution time.
%
After a certain number of generations, the process stops and the best \gls{gene}
is selected.
%
Certain steps are also followed by a routine based on \glsdesc{CP} that prunes
the search space for the subsequent decisions by removing values which will
never appear in any valid \gls{gene}.
%
Although every \gls{gene} represents a single \gls{node} in the \gls{block DAG},
\gls{complex.p} \glspl{pattern} can still be supported through an additional
variable for selecting the \gls{instruction} type for the \gls{node}.
%
If \glspl{node} with the same \gls{instruction} type have been scheduled to be
executed on the same cycle, then they can be implemented using the same
\gls{instruction} during \gls{assembly code} emission.

\citeauthor{LorenzEtAl:2001} originally developed this technique in order to
reduce power usage of \gls{assembly code} generated for constrained \glspl{DSP},
and later extended the design to also incorporate \gls{instruction compaction}
and \gls{address generation}.
%
Experiments indicate that the technique for a selected set of test cases
resulted in energy savings of \SI{18}{\percent} to \SI{36}{\percent} compared to
a traditional \gls{tree covering}-based \gls{compiler}, and reduced execution
time by up to \SI{51}{\percent}.
%
According to \citeauthor{LorenzEtAl:2001}, the major contribution to this
reduction is due to improved usage of \glspl{register} for common subexpression
values, which in turn leads to less use of power-hungry and long-executing
memory operations.
%
But due to the probabilistic nature of \gls{GA}, optimality cannot be
guaranteed, making it unclear how this technique would fare against other
\gls{DAG covering}-based designs which allow a more exhaustive exploration of
the search space.


\subsection{Extending Trellis Diagrams to DAGs}
\labelSection{dc-trellis-diagrams-dags}

In 1998, \citeauthor{HanonoDevadas:1998}~\cite{HanonoDevadas:1998, Hanono:1999}
proposed a technique that is similar to \citeauthor{Wess:1992}'s use of
\glspl{trellis diagram}, which we discussed in \refAppendix{tree-covering} on
\refPageOfSection{tc-trellis-diagrams-trees}.
%
Implemented in a system called \gls!{Aviv}, \citeauthor{HanonoDevadas:1998}'s
\gls{instruction selector} takes a \gls{block DAG} as input and duplicates each
operation \gls{node} according to the number of functional units in the
\gls{target machine} on which that operation can run.
%
Special \gls!{split.n} and \gls!{transfer.n}[ \glspl{node}] are inserted before
and after each duplicated operation \gls{node} to allows the data flow to
diverge and then reconverge before passing to the next operation \gls{node} in
the \gls{block DAG}.
%
The use of \gls{transfer.n} \glspl{node} also allow the cost of transferring
data from one functional unit to another to be taken into account.
%
Similarly to the \gls{trellis diagram}, \gls{instruction selection} is thus
transformed to finding a path from the \gls{leaf} \glspl{node} in the \gls{block
  DAG} to its \gls{root} \gls{node}.
%
But differently from the \gls{optimal.ps}, \mbox{\gls{DP}-oriented} design of
\citeauthor{Wess:1992}, \citeauthor{HanonoDevadas:1998} applied a greedy
heuristic that starts from the \gls{root} \gls{node} and makes it way towards
the \glspl{leaf}.

Unfortunately, as in \citeauthor{Wess:1992}'s design, this technique assumes a
\mbox{1-to-1} mapping between the \glspl{node} in the \gls{block DAG} and the
\glspl{instruction} in order to generate efficient \gls{assembly code}.
%
In fact, the main purpose behind \gls{Aviv} was to generate efficient
\gls{assembly code} for \gls{VLIW} architectures, where the focus is on
executing as many \glspl{instruction} as possible in parallel.


\subsection{Hardware Modeling Techniques}
\labelSection{dc-modeling-entire-target-machines}

In 1984, \textcite{Marwedel:1984} developed a retargetable system called
\gls!{MSS} for \gls{microcode generation},\!%
%
\footnote{%
  \Gls!{microcode} is essentially the hardware language that processors use
  internally for executing \glspl{instruction}.
  %
  For example, \gls{microcode} controls how the \glspl{register} and program
  counter should be updated for a given \gls{instruction}.%
}
%
where a \gls{machine description} written in
\gls!{MIMOLA}~\cite{Zimmermann:1979} is used for modeling the entire data path
of the processor, instead of just the \gls{instruction set} as we have commonly
seen.
%
This is commonly used for \glspl{DSP} where the processor is small but highly
irregular.
%
Although \gls{MSS} consists of several tools, we will concentrate on the
\gls{compiler} -- \gls!{MSSQ} -- as its purpose is most aligned with
\gls{instruction selection}.
%
\gls{MSSQ} was developed by \textcite{LeupersMarwedel:1998} as a faster version
of \gls!{MSSC}~\cite{NowakMarwedel:1989}, which in turn is an extension of the
\gls{tree}-based \gls!{MSSV}~\cite{Marwedel:1993}.

The \gls{MIMOLA} specification contains the processor \glspl{register} as well
as all the operations that can be performed on these \glspl{register} within a
single cycle.
%
From this specification, a hardware~\gls{DAG} called the \gls!{CO graph} is
automatically derived.
%
An example is given in \refFigure{co-graph-example}.
%
\begin{figure}
  \centering%
  % Copyright (c) 2017-2018, Gabriel Hjort Blindell <ghb@kth.se>
%
% This work is licensed under a Creative Commons Attribution-NoDerivatives 4.0
% International License (see LICENSE file or visit
% <http://creativecommons.org/licenses/by-nc-nd/4.0/> for details).
%
\begingroup%
\figureFont\figureFontSize%
\setlength{\nodeSize}{4pt}%
\setlength{\nodeDist}{16pt}%
\begin{tikzpicture}[
    node/.append style={
      fill=black,
    },
    label/.style={
      nothing,
      outer sep=2pt,
      node distance=1mm,
    },
    dots label/.style={
      label,
      outer sep=1mm,
      node distance=0.5\nodeDist,
    },
    flow/.style={
      data flow,
      -
    },
  ]

  % Nodes
  \node [node] (root) {};
  \node [node, node distance=3.5\nodeDist, position=165 degrees from root] (r0)
        {};
  \node [node, node distance=3.5\nodeDist, position=15  degrees from root] (r1)
        {};
  \foreach \i in {0, 1} {%
    \node [node, position=120 degrees from r\i] (r\i load) {};
    \node [node, position=60  degrees from r\i] (r\i noload) {};
    \node [node, position=120 degrees from r\i load] (r\i loadi) {};
    \node [node, position=60  degrees from r\i load] (r\i loadct) {};
    \node [node, position=60  degrees from r\i noload] (r\i noloadct) {};
    \node [node, above=1.5\nodeDist of r\i loadct] (r\i loadctif) {};
    \node [node, right=2\nodeDist of r\i loadctif] (r\i noloadctif) {};
  }
  \node [node] at ($(r0) !.5! (r1)$) (r2) {};
  \node [node, above=2\nodeDist of r2 |- r0noloadctif] (alu) {};
  \node [node, node distance=8\nodeDist, position=170 degrees from alu]
        (viaplus) {};
  \node [node, node distance=8\nodeDist, position=10  degrees from alu]
        (dat) {};
  \node [node] at ($(viaplus) !.33! (dat)$) (plus) {};
  \node [node] at ($(viaplus) !.66! (dat)$) (sub) {};
  \foreach \i in {viaplus, plus, sub, dat} {%
    \node [node, node distance=1.5\nodeDist, position=40  degrees from \i]
          (\i ct) {};
    \ifthenelse{\equal{\i}{sub}}{%
      \node [node, node distance=1.5\nodeDist, position=140 degrees from \i]
            (\i i2) {};
    }{%
      \node [node, node distance=1.5\nodeDist, position=140 degrees from \i]
            (\i i1) {};
    }
    \ifthenelse{\equal{\i}{dat}}{%
      % Add nothing
    }{%
      \ifthenelse{\equal{\i}{sub}}{%
        \node [node] at ($(\i i2) !.5! (\i ct)$) (\i i1) {};
      }{%
        \node [node] at ($(\i i1) !.5! (\i ct)$) (\i i2) {};
      }
    }
    \node [node, above=1.5\nodeDist of \i ct] (\i ctif) {};
  }
  \node [node, above=2.5\nodeDist of plusi2] (top r1) {};
  \node [node, above=3.5\nodeDist of plusi1] (mux) {};

  % Labels
  \node [dots label, above=of r2] (r2 dots) {$\ldots$};
  \node [dots label, above=of mux] (mux dots) {$\ldots$};
  \node [label, left=of r0] {R0};
  \node [label, right=of r2] {R2};
  \node [label, right=of r1] {R1};
  \foreach \i in {r0, r1} {%
    \node [label, left=of \i load] {load};
    \node [label, right=of \i noload] {noload};
    \node [label, left=of \i loadi] {i};
    \node [label, right=of \i loadct] {ct};
    \node [label, right=of \i noloadct] {ct};

  }
  \foreach \i in {0, 1} {%
    \node [label, right=of r\i loadctif] {I.f.R\i};
    \node [label, right=of r\i noloadctif] {I.f.R\i};
  }
  \node [label, right=of alu, yshift=-1mm] {ALU};
  \node [label, left=of viaplus] {via($+$)};
  \node [label, left=of plus] {$+$};
  \node [label, right=of sub] {$-$};
  \node [label, right=of dat] {dat};
  \foreach \i in {viaplus, plus, sub, dat} {%
    \node [label, right=of \i i1] {i1};
    \ifthenelse{\equal{\i}{dat}}{%
      % Add nothing
    }{%
      \node [label, right=of \i i2] {i2};
    }
    \node [label, right=of \i ct] {ct};
    \node [label, right=of \i ctif] {I.f.ALU};
  }
  \node [label, above=of top r1] {R1};
  \node [label, left=of mux, yshift=-2mm] {MUX};

  % Edges
  \begin{scope}[flow]
    \foreach \i in {0, 1, 2} {%
      \draw (root) -- (r\i);
    }
    \foreach \i in {0, 1} {%
      \draw (r\i) -- (r\i load);
      \draw (r\i) -- (r\i noload);
      \draw (r\i load) -- (r\i loadi);
      \draw (r\i load) -- (r\i loadct);
      \draw (r\i noload) -- (r\i noloadct);
      \draw (r\i loadct) -- (r\i loadctif);
      \draw (r\i noloadct) -- (r\i noloadctif);
    }
    \draw (r0loadi) -- (r0loadi |- r0loadctif) -- (alu);
    \draw (r1loadi) -- (alu);
    \foreach \i in {viaplus, plus, sub, dat} {%
      \draw (alu) -- (\i);
      \draw (\i) -- (\i i1);
      \draw (\i) -- (\i ct);
      \ifthenelse{\equal{\i}{dat}}{%
        % Draw nothing
      }{%
        \draw (\i) -- (\i i2);
      }
      \draw (\i ct) -- (\i ctif);
    }
    \foreach \i in {viaplus, plus, sub, dat} {%
      \ifthenelse{\equal{\i}{viaplus}}{%
        \coordinate (halfmark) at ([xshift=\nodeDist] \i i1 |- mux);
      }{%
        \ifthenelse{\equal{\i}{plus}}{%
          \coordinate (halfmark) at (mux);
        }{%
          \coordinate (halfmark) at ([xshift=-\nodeDist] \i i1 |- mux);
        }
      }
      \draw (\i i1) -- ([yshift=\nodeDist] \i i1 |- viaplusctif) -- (halfmark)
        -- (mux);
    }
    \foreach \i in {viaplus, plus, sub} {%
      \ifthenelse{\equal{\i}{viaplus}}{%
        \coordinate (halfmark) at ([xshift=\nodeDist] \i i2 |- top r1);
      }{%
        \ifthenelse{\equal{\i}{plus}}{%
          \coordinate (halfmark) at (top r1);
        }{%
          \coordinate (halfmark) at ([xshift=-\nodeDist] \i i2 |- top r1);
        }
      }
      \draw (\i i2) -- (\i i2 |- viaplusctif) -- (halfmark) -- (top r1);
    }
    \draw (r2) -- (r2 dots);
    \draw (mux) -- (mux dots);
  \end{scope}
\end{tikzpicture}%
\endgroup%


  \caption[Example of a CO graph]%
          {%
            Example of a CO graph for a simple
            processor~\cite{NowakMarwedel:1989}, containing an arithmetic logic
            unit, two data registers, a program counter, and a control store%
          }
  \labelFigure{co-graph-example}
\end{figure}
%
A \gls{pattern matcher} then attempts to find \glspl{subgraph} within the
\gls{CO graph} to cover the \glspl{expression tree}.
%
Because the \gls{CO graph} contains explicit \glspl{node} for every
\gls{register}, a \gls{match} found on this \gls{graph} -- called a
\gls!{version} -- is also an assignment of \gls{function} variables (and
\glspl{temporary}) to \glspl{register}.
%
If a \gls{match} cannot be found (due to a lack of \glspl{register}), the
\gls{expression tree} will be rewritten by splitting assignments and inserting
additional \glspl{temporary}.
%
The process then backtracks and repeats in a recursive fashion until the entire
\gls{expression tree} is covered.
%
A subsequent process then selects a specific \gls{version} from each \gls{match
  set} and tries to schedule them so that they can be combined into
\glspl{bundle} for parallel execution.

Although \gls{microcode generation} is at a lower hardware level than
\gls{assembly code} generation -- which is usually what we refer to with
\gls{instruction selection} -- we see several similarities between the problems
that must be solved in each.
%
For this reason it is included in this survey, and further examples
include~\cite{BalakrishnanEtAl:1986, MahmoodEtAl:1990, LangevinCerny:1993}.
%
In \refAppendix{graph-covering}, we will see another design that also models the
entire processor but applies a more powerful technique.


\section{Limitations of DAG Covering}
\labelSection{dc-limitations}

Although \glspl{DAG covering} addresses the issue of whether to \glsshort{edge
  splitting} or \glsshort{node duplication}[e] common subexpressions within a
\gls{block}, the problem still remains for expressions that are spread across
multiple \glspl{block}.
%
To fully address this problem, one must resort to \gls{graph covering}.

This also applies to other situations where decisions made for one \gls{block}
can inhibit subsequent decisions for other \glspl{block}, such as enforcing
specific storage locations or value modes.
%
For example, \refFigure{block-dags-limit-example-2} shows a \gls{function} that
multiplies the elements of two arrays and sums the results.
%
\begin{filecontents*}{block-dags-limit-example.c}
int f(int* A, int* B, int N) {
  int s = 0;
  for (int i = 0; i < N; i++) {
    s = s + A[i] * B[i];
  }
  return s;
}
\end{filecontents*}
%
\begin{figure}
  \centering%
  \subcaptionbox{C code\labelFigure{block-dags-limit-example-2-c}}%
                {%
                  \begin{lstpage}{56mm}%
                    \lstinputlisting[language=c]{block-dags-limit-example.c}%
                  \end{lstpage}%
                }%
  \hfill%
  \subcaptionbox{%
                  Block graphs involving variable~\irVar*{s}.
                  %
                  For brevity, the subtrees concerning \irCode*{A[i]}
                  and \irCode*{B[i]} are not included%
                  \labelFigure{block-dags-limit-example-2-dags}%
                }%
                [60mm]%
                {%
                  ../existing-isel-techniques-and-reps/block-dags-limit-example-dags.tex%
                }

  \vspace{\betweensubfigures}

  \subcaptionbox%
    {%
      Rules.
      %
      For brevity, the actions are not included.
      %
      $\mNT{Null}$ is a dummy nonterminal since \irCode*{\irRetText} does not
      return anything, yet all productions must have a result.
      %
      All rules are assumed to have equal cost%
    }%
    [\textwidth]%
    {%
      \figureFontSize%
      \newcolumntype{L}{@{}l@{}}%
      \begin{tabular}{r@{ $\rightarrow$ }l@{\hspace{3em}}r@{ $\rightarrow$ }lc}
        \toprule
        \multicolumn{5}{c}{\tabhead rules}\\
        \midrule
        $\mNT{Reg}$ & \irCode{const}
          & $\mNT{SReg}$
          & \multicolumn{2}{L}{%
              $\irCode{\irMulText} \; \mNT{Reg} \; \mNT{Reg}$%
            }\\
        $\mNT{SReg}$ & \irCode{const}
          & $\mNT{Null}$ & \multicolumn{2}{L}{%
              $\irCode{\irRetText} \; \mNT{Reg}$%
            }\\
        $\mNT{Reg}$ & $\irCode{\irAddText} \; \mNT{Reg} \; \mNT{Reg}$
          & $\mNT{Reg}$  & $\mNT{SReg}$ & $(r \ll 1)$\\
        $\mNT{SReg}$ & $\irCode{\irAddText} \; \mNT{SReg} \; \mNT{SReg}$
          & $\mNT{SReg}$ & $\mNT{Reg}$  & $(r \gg 1)$\\
        \bottomrule
      \end{tabular}%
    }

  \caption{Example illustrating the limitation of block DAGs}
  \labelFigure{block-dags-limit-example-2}
\end{figure}
%
Assume that the arrays consist of fixed-point values.
%
For efficiency, a common idiosyncrasy in many \glspl{DSP} is that multiplication
of two fixed-point values return a value that is shifted one bit to the left.
%
For such \glspl{target machine}, both the value~\irCode*{0} and the accumulator
\gls{variable}~\irVar*{s} should be in shifted mode throughout the entire
\gls{function}, and only restored into normal mode upon return.
%
Otherwise the accumulated value would be needlessly shifted back and forth
within the loop.
%
Achieving this, however, is difficult when limited to covering only a single
\gls{block DAG} at a time.
%
Assume for example that the function had no multiplication.
%
In that case, deciding to load value~\irCode*{0} in shifted mode would instead
lower code quality as the value would needlessly have to be shifted back before
returning, which takes an extra \gls{instruction}.

Lastly, most of these approaches are restricted to tree-shaped \glspl{pattern},
meaning they only support \gls{single-output.ic} \glspl{instruction}.
%
Many \glspl{instruction set}, however, contain \gls{multi-output.ic}
\glspl{instruction} and require \gls{DAG}-shaped \glspl{pattern}, which violate
underlying assumptions made by many of the aforementioned approaches.


\section{Summary}
\labelSection{dc-summary}

In this appendix, we have investigated several methods that rely on the
\gls{principle} of \gls{DAG covering}, which is a more general form of \gls{tree
  covering}.
%
Operating on \glspl{DAG} instead of \glspl{tree} has several advantages.
%
Most importantly, common subexpressions can be directly modeled, and a larger
set of \glspl{instruction} -- including \gls{multi-output.ic} and
\gls{disjoint-output.ic} \glspl{instruction} -- can be supported and exploited
during \gls{instruction selection}.
%
This in turn leads to improved performance and reduced code size.
%
Consequently, techniques based on \gls{DAG covering} are today one of the most
widely applied methods for \gls{instruction selection} in modern
\glspl{compiler}.

The ultimate cost of transitioning from \glspl{tree} to \glspl{DAG}, however, is
that \gls{optimal.ps} \gls{pattern selection} can no longer be achieved in
linear time as it is NP-complete.
%
At the same time, \glspl{DAG} are not expressive enough to allow the proper
modeling of all aspects featured in the \glspl{function} and
\glspl{instruction}.
%
For example, statements such as \mbox{for loops} incur \glspl{loop} in the
\gls{graph} representing the \gls{function}, restricting \gls{DAG covering} to
the scope of \glspl{block} and excluding the modeling of \gls{inter-block.ic}
\glspl{instruction}.
%
Another disadvantage is that optimization opportunities for storing
\gls{function} variables and \glspl{temporary} in different forms and at
different locations across the \gls{function} are forfeited.

In the next appendix, we will discuss the last and most general \gls{principle}
of \gls{instruction selection}, which addresses some of the aforementioned
deficiencies of \gls{DAG covering}.
